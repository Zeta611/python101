\documentclass[../main.tex]{subfiles}

\begin{document}

\subsection{강의 계획}
본 교재에는 Python 2.7 까다로울 수 있는 숙제 문제들과 직접 제작하거나 다양한 자료에서 모은 예제들이 수록되어 있습니다.
강의 수준은 기초부터 심화까지 차근차근 진행할 것이며, 상황에 따라 적절히 조절하도록 하겠습니다.
본 교재에서 다루는 내용은 어렵고 복잡한 algorithm알고리즘을 배우는 것이라기보다는, Python 2 언어의 기본적인 문법 숙지와 더불어 자주 사용되는 패턴에 익숙해지는 것에 초점을 맞추고 있습니다.
또한, 정보과학적인 사고력(순차적으로 논리적인 비약이 없이 문제 해결을 할 수 있는 능력)를 배우고 초보적인 algoritnhmic알고리즈믹 문제의 해결법을 배우는 것이 목적입니다.
본문의 내용 중 Toy Robot 관련 내용은 수업을 진행하지 않을 예정으로, 한국과학영재학교의 교육 과정에 포함되어 있는 내용입니다.
아래의 진도표는 회차 당 세 시간 수업을 기준으로 한 것이며, 다른 수업 계획이나 학생의 이해도에 따라 유동적일 수 있습니다.

\begin{description}
    \item[1회차] Python 소개 및 기본 요소
    \item[2회차] Functions함수와 Conditionals조건문
    \item[3회차] Boolean Functions불리언 함수와 Loops반복문 기본
    \item[4회차] Lists리스트, Strings문자열, Counters카운터
    \item[5회차] Quantifiers한정자와 While 문
    \item[6회차] Loops반복문 응용과 파일 입출력
    \item[\sout{7회차}] \sout{Toy Robot 활용과 기출 문제 풀이}
    \item[7회차] 재귀법Recursion과 Python의 다양한 객체
    \item[8회차] 람다Lambda 함수
\end{description}

\subsection{들어가기 전에}
Python은 문법이 단순하면서도 활용 가능성이 무궁무진한 언어입니다.
단순히 `정보과학' 과목만을 위해서 Python을 배운다기보다는, 자신이 평생 쓸 수 있는 도구를 연습한다는 마음가짐으로 배웠으면 좋겠습니다.
Python은 R이나 Matlab 등과 함께 학문적인 용도로도 쓰임이 많은 언어입니다.
특히 Matlab과 다르게 Python은 오픈 소스에 무료인데다가, 단순히 데이터 처리만을 위한 언어가 아니기 때문에 유연합니다.
NumPy와 Matplotlib라는 패키지로 Matlab이 할 수 있는 다양한 작업들을 똑같이 할 수 있기도 합니다.

실제로 저는 한국과학영재학교에서 정보과학 과목 이외에도 일반물리학실험이나 일반천문학실험, 일반화학실험 등의 과목에서 데이터를 분석하고 차트를 그리기 위해 Python을 사용하였습니다.
나아가 R\&E와 같은 연구 활동이나 KYPT와 같은 대회에 참가한다면 Python을 통한 데이터 분석 및 처리를 통해서 분명 큰 도움을 받을 수 있습니다.

\subsection{Python이란?}
Python은 1991년에 Guido van Rossum이 발표한 프로그래밍 언어로, 문법이 굉장히 쉬우면서도 높은 생산성을 가지고 있는 언어입니다.
특히 pseudocode의사코드와 거의 형태가 유사하기 때문에 배우기 쉬워서, 많은 학교에서 프로그래밍 입문 수업 언어로 Python을 채택하고 있습니다.
또, \textbf{프로그래밍 언어로 할 수 있을 법한 거의 모든 기능들이 이미 Python을 위한 패키지로 구현되어 있습니다.}
대부분의 연장이 미리 준비되어 있는 것입니다.
따라서 속도가 중요한 작업이 아니라면 C/C++보다 Python을 쓰는 것이 훨씬 효율적입니다. (연산이 아니라 업무의 효율을 말하는 것입니다.)

프로그래밍에 익숙치 않더라도 컴퓨터에 관심이 많다면 C, C++, Java 등의 언어와 함께 Python도 한 번 쯤 들어보았을 정도로, Python은 점유율 상위 다섯 언어 안에 드는 주류 언어입니다.
Python은 1995년에 등장한 Java보다도 오래 전에 만들어진 언어로, 긴 역사를 가지고 있습니다.
그 만큼 버전의 숫자도 큰데, 이 글을 작성하는 현 시점에서 Python 2의 최신 버전은 2.7.14, Python 3의 최신 버전은 3.6.4입니다.
그런데 왜 Python 2와 Python 3를 둘 다 언급하냐고요?
아직까지도 호환성 문제로 인해서 Python 2를 사용하는 경우가 많기 때문입니다.
한동안은 Python 2와 Python 3가 동시에 개발되기도 하였습니다.
물론, 지금 처음 프로그래밍을 배우시는 분이라면 이제는 Python 2보다는 Python 3를 추천하는 경우가 잦아졌습니다.
Python 2는 2.7이 최종 버전이고, 이후에는 bugfix 릴리즈만 있을 예정인데다가 2020년까지만 지원을 할 예정입니다.
나아가 최신 버전의 Ubuntu(Linux 배포판의 한 버전)는 Python 3를 기본 Python 버전으로 설정하였습니다.
1, 2년 전까지만해도 Python 2 vs Python 3의 논쟁이 비일비재하게 일어났었습니다.
그렇지만 우리 학교에서는 Python 2를 배우는 탓에, 나아가 cs1graphics 등의 모듈과의 호환성 때문에 Python 2.7.6을 사용해야 합니다.

\textbf{Python은 interpreter인터프리터 언어입니다.}
Interpreter 언어란 소스 코드를 한 줄씩 기계어\footnote{기계가 바로 이해할 수 있는 저급 언어로, 0과 1의 이진수로만 구성되어 있는 언어로, 모든 CPU를 구동시키기 위해서는 C와 같은 고급 언어를 저급 언어로 변환시키는 과정이 필요합니다.}로 번역하는 방식의 언어입니다.
조금 더 가독성을 높인, 기계어와 일대일 대응이 되는 (마찬가지로 저급 언어인) 어셈블리어가 있기도 합니다.
고급 언어를 기계어로 변환시키는 방법은 크게 두 가지가 있습니다.
Compiler컴파일러와 interpreter입니다.
C와 같은 언어는 compiler 언어로, 코드 전체를 한꺼번에 기계어로 변환시킵니다.
이 때문에 실행 속도는 빠르다는 장점이 있지만, 코드를 수정하기 위해서는 다시 한 번 전체를 기계어로 변환시키는 과정이 필요합니다.
반면 Python은 한 줄씩 기계어로 번역하기 때문에 실행 속도는 다소 느리지만, debug디버그\footnote{코드에서 bug버그, 즉 오류를 제거하는 것을 의미합니다.}에 유리합니다.
한 줄씩 실행하기 때문에 애초에 compile을 거부하는 compiler 언어와는 다르게 버그가 있는 해당 줄까지 코드를 실행시켜주기 때문입니다.
이러한 장단점 때문에 프로그램의 골격을 Python으로 만들고, 빠른 연산이 필요한 부분만 C로 만드는 것도 가능합니다.

마지막으로, Python이 얼마나 쉽고 직관적인 언어인지 알아봅시다.
만약 A+가 F, C+. B0, A-, A+에 있으면 "A+가 있습니다."를 실행해주는 Python 코드를 봅시다.
\begin{minted}[mathescape,
               breaklines,
               numbersep=5pt,
               frame=lines,
               framesep=2mm]{python}
if "A+" in ["F", "C+", "B0", "A-", "A+"]: print "A+가 있습니다."
\end{minted}
프로그래밍을 할 줄 모르더라도 저 코드를 이해할 수 있을 것입니다.
이처럼 Python은 인간이 사고하는 방식을 그대로 옮겨 놓았다고 해도 과언이 아닐만큼이나 직관적입니다.
\textbf{익숙해진다면 자신이 하고 싶은 일을 코드로 옮기려고 끙끙댈 필요 없이, 생각하는 그대로 코드를 작성할 수 있는 자신을 발견할 수 있을 것입니다.}
이제 Python의 기본 요소에 대해 알아봅시다!
(설치 과정은 따로 담지 않았습니다. 수업 중에 같이 진행하여 봅시다.)

\subsection{Python의 기본 요소}
\subsubsection{Values값과 Variables변수}
IDE에 내장된 쉘shell에 다음을 입력해 봅시다.
Interpreter 언어이기 때문에, 명령을 한 줄씩 입력하여 명령을 수행할 수 있습니다.
(\texttt{>>>}는 직접 입력하는 것이 아닙니다.)
\begin{minted}[mathescape,
               linenos,
               breaklines,
               numbersep=5pt,
               frame=lines,
               framesep=2mm]{python}
>>> a = (1+2+3+4)/2
>>> b = a - 1
>>> c = 3
>>> print b + c
7
>>> print b, c
4  3
>>> d = c
>>> a = d
>>> print a
3
\end{minted}
\texttt{a = (1+2+3+4)/2}는 등호 왼쪽에 있는 \texttt{a}에 등호 오른쪽에 있는 $\frac{1+2+3+4}{2}$를 배정한다는 뜻입니다.
즉, 수학에서 말하는 등호와는 의미가 다른 것이지요.
줄 2의 \texttt{b = a - 1}은 현재 \texttt{a}에 배정된 $\frac{1+2+3+4}{2} = 5$의 값에 1을 뺀 4를 \texttt{b}에 배정한다는 뜻입니다.
이 때, 좌변에 있는 \texttt{a}, \texttt{b	}, \texttt{c}를 variables변수\footnote{필요에 따라 영문의 `variable', 국문의 `변수'를 사용하여 서술하겠습니다. 이외의 용어도 처음에는 영문과 국문을 병기한 후, 필요에 따라 둘 중 하나의 표현만 사용하겠습니다.}, 우측의 \texttt{(1+2+3+4)/2}를 expression표현이라고 합니다.
또한 이러한 변수에 배정되는 것을 value값이라고 합니다.
다시 위의 예시로 돌아가서, 줄 2에 있는 좌변의 \texttt{b}는 variable, 우변의 \texttt{a - 1}는 value인 것이죠.
Variable과 value는 문자인지 숫자인지의 여부가 아니라 대입이 되는 대상인지 대입이 되어지는 대상인지의 여부가 결정짓습니다.
물론, variable이 value 그 자체가 될 수 있습니다.
줄 8의 예시와 같은 경우, 우변의 \texttt{c}는 variable이면서 \texttt{d}라는 variable에 대입되는 value입니다.
그렇다면, 아래와 같은 표현을 어떨까요?
\begin{minted}[mathescape,
               breaklines,
               numbersep=5pt,
               frame=lines,
               framesep=2mm]{python}
>>> 10 = a
\end{minted}
지금까지 잘 따라왔다면, value가 위치해야 할 좌변에 value인 literal\footnote{어떤 객체에 value를 부여할 수 있는 것입니다. \texttt{0}은 \texttt{int} literal, \texttt{"python"}은 \texttt{str} literal입니다. \texttt{int}, \texttt{str}이 무엇인지는 조금 뒤에 type형이라는 것을 배우면 알 수 있습니다.}이 위치해 잘못된 syntax구문이라는 것을 알 수 있습니다.
\texttt{SyntaxError: can't assign to literal}이라는 오류 log를 볼 수 있을 것입니다.
(여담이지만, 앞으로 수 없이 많은 오류 log를 볼 텐데, 이를 꼼꼼히 읽어보는 습관을 들입시다.
오류 log를 무시하고 디버깅하는 것은 마치 백사장에서 바늘을 찾는 것과도 같습니다.)
혹은, ``let 10 be a"라는 표현이 말이 되지 않는다는 것을 통해서도 직관적으로 잘못되었음을 파악할 수 있습니다.

이러한 변수는 여러 가지 종류가 있습니다.
이를 data type자료형, 혹은 간단히 type형이라고 합니다.
\texttt{-1}, \texttt{0}, \texttt{76} 등의 값은 \texttt{int}\footnote{정수라는 뜻의 integer에서 따온 명칭입니다. 단, 수학에서 3.0은 정수이지만 \texttt{3.0}은 \texttt{float} type입니다.} type, \texttt{3.14159}, \texttt{0.}, \texttt{6.626e-34}, \texttt{6.022E23} 등의 값은 \texttt{float}\footnote{부동 소수점 혹은 떠돌이 소수점이라는 뜻의 floating point에서 따온 명칭입니다. 실수가 아니라 근삿값이라는 의미에 더 가깝습니다.} type, \texttt{"Hello, world!"}, \texttt{\textquotesingle python\textquotesingle}, \texttt{""} 등의 값은 \texttt{str}\footnote{나열이라는 뜻의 string에서 따온 명칭입니다.} type입니다.
Type에 대해서는 조금 뒤에 더 자세히 살펴봅시다.

변수는 어떤 값이 배정된 것이라고 했는데, 이것을 assign되었다고 하며 값이 written되었다고도 표현할 수 있습니다.
이렇게 변수에 저장된 값은 read읽을 수 있는데, 첫 예시의 줄 2처럼 \texttt{a}에 저장된 값 \texttt{5}를 불러오는 것이 이에 해당합니다.
또한 줄 4의 \texttt{print}를 통해서도 값을 읽어올 수 있습니다.

변수는 서로 다른 값이 배정될 수 있습니다.
첫 예시에서 줄 9를 보면, \texttt{5}가 저장되었던 \texttt{a}에 \texttt{3}이 저장된 \texttt{d}의 value가 다시 \texttt{a}에 쓰여지는 것을 볼 수 있습니다.
줄 10에서 이 사실을 재확인할 수 있고요.
이와 같이 변수는 재활용될 수 있고, 이는 개수를 세거나(counting) 특정 사건을 기록하기 위해 flag 등으로 사용하는데 도움이 됩니다.
\begin{minted}[mathescape,
               linenos,
               breaklines,
               numbersep=5pt,
               frame=lines,
               framesep=2mm]{python}
>>> summ = 0
>>> summ = summ + 1
>>> summ = summ + 1
>>> print summ
2
\end{minted}
더 자세한 활용은 차차 Python을 익혀가면서 알아봅시다.

이 뿐만이 아니라, Python은 여러 변수에 여러 값을 한 번에 배정하는 것(multiple assignment)을 허용합니다.
나아가 \textbf{변수의 swapping을 매우 손쉽게 할 수 있습니다.}
이 둘을 다음 예시를 통해 함께 확인합시다.
\begin{minted}[mathescape,
               linenos,
               breaklines,
               numbersep=5pt,
               frame=lines,
               framesep=2mm]{python}
>>> a, b, c = 2, 7, 12
>>> a, b, c = b, c, a
>>> print a, b, c
(7, 12, 2)
>>> a, b = b % a, a
>>> print a, b
5, 7
\end{minted}
Multiple assignment를 허용하지 않는 대다수의 언어에서는 임시 변수를 도입해야 합니다.
예컨대 줄 5의 경우 아래와 같은 방법을 사용해야 합니다.
\begin{minted}[mathescape,
               linenos,
               breaklines,
               numbersep=5pt,
               frame=lines,
               framesep=2mm]{python}
>>> tmp = a
>>> a = b % tmp
>>> b = tmp
\end{minted}
Python에서는 마치 하노이의 탑을 연상시키는 이러한 과정을 시행하지 않아도 됩니다.

마지막으로, 변수의 이름으로 정할 수 없는 특정 문자열이 있습니다.
Python이 내부적으로 사용하는 \texttt{int}, \texttt{str}, \texttt{if}, \texttt{else}, \texttt{for}, \texttt{range}, \dots 등이 이에 해당합니다.
(제가 확인해본 결과 \texttt{int} 등 type 명은 배정이 가능했으나, 굳이 그렇게 명명을 해서 좋을 것은 없을 것 같습니다.)
그리고 변수명은 영어 대소문자, 숫자, 그리고 \texttt{\_}로만 이뤄져 있어야 합니다.\footnote{Python 3에서는 한글로도 이름을 명명할 수 있다고 합니다.}
나아가 숫자로 시작할 수 없습니다.
\texttt{1st\_name}과 같은 문자열을 변수명으로 지정할 수 없는 것입니다.

\subsubsection{Expressions표현}
Expression은 variable, value, 그리고 operator들의 조합입니다.
우리가 흔히 부르는 사칙 연산 \texttt{+}, \texttt{-}, \texttt{*}, \texttt{/}와, 나머지를 구해주는 \texttt{\%}, 지수를 뜻하는 \texttt{**} 등이 이에 해당합니다.
\textbf{주의할 점은, \texttt{\^{}}이 지수를 뜻하는 것이 아니라 \texttt{**}이라 것입니다.}
또한, \texttt{/}는 정수 사이의 연산에서는 몫을 구해줍니다.
아래의 예시를 봅시다.
\begin{minted}[mathescape,
               linenos,
               breaklines,
               numbersep=5pt,
               frame=lines,
               framesep=2mm]{python}
>>> 12 + 5
17
>>> 12 - 5
7
>>> 12 * 5
60
>>> 12 / 5
2
>>> 12.0 / 5.0, 12. / 5., 12. / 5, 12 / 5.
2.4  2.4  2.4  2.4
>>> 12 % 5
2
>>> 12 ** 5
248832
>>> 12 ^ 5
9
\end{minted}
\texttt{\^{}}는 bitwise XOR의 연산자로서, $12 = 1100_2$, $5 = (0)101_2$이므로 digit이 다른 $2^3, 2^2, 2^1$ 자릿수만 1을 취한 $1001_2 = 9$가 \texttt{12 \^{} 5}의 값입니다.
하지만 우리는 bitwise XOR이 무엇인지 알 필요는 없습니다.
\texttt{\^{}}이 우리가 원하는 지수의 연산이 아니라는 사실만 기억해두면 됩니다!
연산의 순서는 기본적으로 괄호(\texttt{(\dots)}), unary 연산(\texttt{+x}, \texttt{-x}), 지수(\texttt{**}), 곱셈/나눗셈/나머지 연산(\texttt{*}, \texttt{/}, \texttt{\%}), 덧셈/뺄셈(\texttt{+}, \texttt{-})의 순서입니다.
억지로 외울 필요는 없고, 연습을 하다 보면 자연스레 익힐 수 있을 것입니다.
헷갈리는 경우에는 \texttt{(\dots)}를 사용하여 순서를 덮어 씌울 수 있습니다.

나누기 연산은 정수끼리 행하면 몫 만을 구해준다고 하였습니다.
그런데 줄 9에서는 수학적으로는 정수끼리 행하였지만 여전히 올바른 실수의 값, 2.4를 내놓는 것을 볼 수 있습니다.
이는 \texttt{12.}이나 \texttt{5.}이라는 literal 자체가 \texttt{float} type이라는 것을 표현하기 때문입니다.
\textbf{이러한 나누기 연산에서 제수와 피제수 둘 중 하나만이라도 type이 \texttt{float}이면 결과값도 \texttt{float}을 가지게 됩니다.}\footnote{Python 3에서는 이와 상관 없이 \texttt{float} type의 정확한 나눗셈 결과를 반환합니다.}
이 부분도 자주 실수를 하게 되는 부분이니 꼭 주의하시길 바랍니다.

코드를 작성할 때에는 특별한 경우를 제외하고는 가독성이 중요합니다.
예컨대, 중복되는 값이나 의미가 있는 값은 특정 변수에 저장하여 해당 변수를 통해 식을 표현하는 것이 바람직합니다.
아래의 예시를 봅시다.
\begin{minted}[mathescape,
               linenos,
               breaklines,
               numbersep=5pt,
               frame=lines,
               framesep=2mm]{python}
>>> S = ((3 + 4 + 5) * (-3 + 4 + 5) * (3 - 4 + 5) * (3 + 4 - 5))**0.5
>>> a, b, c = 3, 4, 5
>>> s = (a + b + c) / 2.
>>> S = (s * (s - a) * (s - b) * (s - c))**0.5
\end{minted}
비록 줄 수는 늘어났지만, 줄 1의 표현보다는 줄 4의 표현이 가독성이 높을 뿐만이 아니라 더 일반적이어서 값을 바꾸기 위해서는 줄 2의 숫자 부분만 변경을 하면 된다.
반면 줄 1의 표현의 경우 \texttt{+}와 \texttt{-}의 부호 구분에서 실수를 할 수 있고 다른 값을 대입하기 위해서는 12 부분에 수정을 가해야 한다.
그리고 다시 강조하지만, 줄 3처럼 나눗셈 연산을 할 때 정확한 값을 얻기 위해서 제수와 피제수 중 하나는 \texttt{float} type이어야 합니다.

마지막으로 소개할 syntax는 위에서 잠시 언급한 개수 세기 등에서 유용하게 쓸 수 있습니다.
변수 뒤에 산술 연산자(\texttt{+}, \texttt{-}, \texttt{*}, \texttt{/}, \texttt{\%}, \texttt{**}) 뒤에 바로 \texttt{=}를 붙인 후 수를 쓰는 syntax입니다.\footnote{\texttt{int}나 \texttt{float}형에서는 모든 산술 연산자를 쓸 수 있고, \texttt{str}형에 대해서는 정의가 되어 있는 \texttt{+}만 사용할 수 있습니다.}
\texttt{x += 1}과 같이 말입니다.
이는 해당 변수에 저장된 값에 등호 뒤에 쓰인 값을 더한다는 의미로, \texttt{x = x + 1}과 동일한 의미를 가지고 있습니다.
아래와 같이 응용할 수 있습니다.
\begin{minted}[mathescape,
               linenos,
               breaklines,
               numbersep=5pt,
               frame=lines,
               framesep=2mm]{python}
>>> x = 4
>>> x += 2
>>> x
6
>>> x -= 1
>>> x
5
>>> x *= 2
>>> x
10
>>> x /= 5
>>> x
2
>>> x %= 3
>>> x
2
>>> x **= 3
>>> x
8
\end{minted}

\subsubsection{Types형}
위에서 간단히 소개한 바 있는데, variable의 종류를 type형이라고 합니다.
현재로서는 지금까지 언급한 \texttt{int}(정수형), \texttt{float}(실수형), \texttt{str}(문자열) 세 가지 type만 알아두시면 됩니다.
소리 내어 읽을 때 저는 보통 인트, 플로트, 스트링과 같이 읽습니다.
지금까지는 숫자가 실수형임을 명시할 때 \texttt{3.}과 같이 온점을 찍어 표현하였는데, type conversion형 변환이라는 것을 사용하여도 됩니다.
\textbf{형 변환은 정수형과 실수형 간에서 자유롭게 가능하고, 문자열의 경우에는 그 자체가 수일 경우에만 변환이 가능합니다.}
\begin{minted}[mathescape,
               linenos,
               breaklines,
               numbersep=5pt,
               frame=lines,
               framesep=2mm]{python}
>>> x = 76
>>> x
76
>>> float(x)
76.0
>>> str(x)
'76'
>>> pi = 3.14
>>> pi
3.14
>>> int(pi)
3
>>> str(pi)
'3.14'
>>> s = "1"
>>> s
'1'
>>> int(s)
1
>>> float(s)
1.0
\end{minted}
위 예시를 통해 필요한 모든 경우를 파악하셨을 것입니다.
또한, \texttt{type($\cdot$)}를 통해 직접 type을 확인할 수 있습니다.
\begin{minted}[mathescape,
               linenos,
               breaklines,
               numbersep=5pt,
               frame=lines,
               framesep=2mm]{python}
>>> print type(76)
<type 'int'>
>>> print type(76.)
<type 'float'>
>>> print type("76")
<type 'str'>
>>> print type(76/1)
<type 'int'>
>>> print type(76./1)
<type 'float'>
\end{minted}

\subsubsection{Input/Output입출력}
지금까지는 Python shell에서만 명령을 실행했습니다.
그렇기 때문에--에컨대--\texttt{a}에 담긴 값을 알기 위해서는 굳이 \texttt{print a}가 아니라 \texttt{a}를 치는 것 만으로도 충분했습니다.
하지만 여러 줄의 코드를 한꺼번에 작성하여 실행할 때에는 이런 방식의 접근이 불가능합니다.
또, 코드를 실행 중일 때 어떤 입력을 받기 위해서는 지금까지와는 다른 방법이 필요합니다.
Shell과는 다르게 한 줄씩 직접 입력하는 방식이 아니기 때문입니다.
값을 출력하는 것은 지금까지 해왔던 것처럼 \texttt{print}를 사용하면 되는데, 아래에서 \texttt{print}에 대해 좀 더 알아보겠습니다.
Shell이 아니라 파일을 만들어서 실행시킵니다.
\begin{minted}[mathescape,
               linenos,
               breaklines,
               numbersep=5pt,
               frame=lines,
               framesep=2mm]{python}
print "Hello, world!"

print 1
print 2
print 3

print 1,
print 2 ,
print 3,
print 

print 1, 2, 3

today = "Thursday"
print "Today is ", today
\end{minted}
위 코드를 실행시키면, \texttt{print 1}과 같이 뒤에 쉼표 \texttt{,}를 붙이지 않으면 줄이 바뀌고, \texttt{,}를 붙이면 줄이 바뀌지 않는다는 것을 알 수 있습니다.
또한 줄 8에서 보는 것처럼 값 뒤에 띄어쓰기를 한 후 쉼표를 붙여도 마찬가지로 줄이 바뀌지 않습니다.
줄 10에는 \texttt{print}만 단독으로 나와 있는데, 이는 출력문 없이 줄만 바뀌는 효과를 줍니다.
줄 7에서 10까지의 코드는 줄 12와 동등한 결과를 출력합니다.
줄 15처럼 type이 다르더라도 섞어서 출력할 수 있습니다.

이제는 값을 입력하는 법에 대해 알아봅시다.
\texttt{raw\_input($\cdot$)} 함수를 사용하면 됩니다.
다음과 같은 예시를 살펴봅시다.
\begin{minted}[mathescape,
               linenos,
               breaklines,
               numbersep=5pt,
               frame=lines,
               framesep=2mm]{python}
today = raw_input("What day is it today? ")
print "Today is ", today

s = raw_input("Enter the number you want to know the square root of: ")
n = float(s)
print n**.5
\end{minted}
위 코드를 실행시키면 창에 \texttt{What day is it today? }가 출력된 후 입력이 될 때까지 기다립니다.
키보드로 값을 입력한 후--\texttt{Thursday}를 입력했다고 합시다--enter 키를 치면 값이 입력되고, \texttt{Today is Thursday.}가 출력될 것입니다.
또한, 수를 입력 받을시 \texttt{int($\cdot$)}나 \texttt{float($\cdot$)}로 형 변환을 수행해줘야 합니다.
\textbf{\texttt{raw\_input($\cdot$)}이 넘겨주는 값은 항상 \texttt{str} 형이기 때문입니다.}

\subsection{예제}
\begin{enumerate}
\item Add additional code to the following so that the difference between the sum of the squares of the first $n$ natural numbers and the sum of the cubes is printed.
Note that \texttt{for} statement is not needed; just use formulae for the sums of the squares and the cubes.
\begin{minted}[mathescape,
               linenos,
               breaklines,
               numbersep=5pt,
               frame=lines,
               framesep=2mm]{python}
n = int(raw_input("Enter n: "))
# Add here!
\end{minted}

\item There are five people sitting around a table: \texttt{a}, \texttt{b}, \texttt{c}, \texttt{d}, and \texttt{e}.
Each person writes down his/her favorite integer.
Then each person adds the numbers the two people sitting next to him/her, to the original number he/she wrote.
The following code should print the numbers \texttt{a}, \texttt{b}, \texttt{c}, \texttt{d}, and \texttt{e} achieved at the end.
Add additional code to make the following code work.
\begin{minted}[mathescape,
               linenos,
               breaklines,
               numbersep=5pt,
               frame=lines,
               framesep=2mm]{python}
a = int(raw_input("Enter a: "))
b = int(raw_input("Enter b: "))
c = int(raw_input("Enter c: "))
d = int(raw_input("Enter d: "))
e = int(raw_input("Enter e: "))
print "Favorite integers: ", a, b, c, d, e
# Add here!
print "Final integers: ", a, b, c, d, e
\end{minted}

\item You want to solve a quadratic equation $\texttt{a}x^2 + \texttt{b}x + \texttt{c} = 0$.
Add additional code to make the following code work.
Assume that $\texttt{b}^2 - 4\texttt{ac} > 0$.
\begin{minted}[mathescape,
               linenos,
               breaklines,
               numbersep=5pt,
               frame=lines,
               framesep=2mm]{python}
a = int(raw_input("Enter a: "))
b = int(raw_input("Enter b: "))
c = int(raw_input("Enter c: "))
# Add here!
# x1 = ...
# x2 = ...
print "Solutions for the quadratic equation are ", x1, " and ", x2
\end{minted}

\item Predict the output of the following expressions.
\begin{minted}[mathescape,
               linenos,
               breaklines,
               numbersep=5pt,
               frame=lines,
               framesep=2mm]{python}
>>> 3*2**6+12
???
>>> 32/7+2**3/3
???
>>> 13/3%3+2.0
???
>>> 13/2/4.
???
>>> 13/2./4
???
>>> float(13/2)/4
???
>>> int(2**(1/2))+1
???
\end{minted}
Choose all possible variable names in Python 2 from the following:
\begin{itemize}
\item \texttt{if}
\item \texttt{sum}
\item \texttt{max}
\item \texttt{1st\_var}
\item \texttt{var\_1}
\item \texttt{this+that}
\item \texttt{\_self}
\item \texttt{변수}
\item \texttt{r\&e}
\end{itemize}

\item The following code reads six real numbers \texttt{x1}, \texttt{y1}, \texttt{x2}, \texttt{y2}, \texttt{x3}, and \texttt{y3}.
Assume that $(\texttt{xi}, \texttt{yi})$ (\texttt{i = 1, 2, 3}) are distinct points in a coordinate plane.
Add additional code so that the code prints out the area of the triangle formed by the three points.
\begin{minted}[mathescape,
               linenos,
               breaklines,
               numbersep=5pt,
               frame=lines,
               framesep=2mm]{python}
x1 = int(raw_input("Enter x1: "))
y1 = int(raw_input("Enter y1: "))
x2 = int(raw_input("Enter x2: "))
y2 = int(raw_input("Enter y2: "))
x3 = int(raw_input("Enter x3: "))
y3 = int(raw_input("Enter y3: "))
# Add here!
# area = ...
print "Area of the triangle is", area
\end{minted}

\item Calculate the remainder of \texttt{a} when divided by \texttt{b}.
Do not use \texttt{\%} but only \texttt{-}, \texttt{+}, and \texttt{/}.
Assume that both \texttt{a} and \texttt{b} are positive integers.
\begin{minted}[mathescape,
               linenos,
               breaklines,
               numbersep=5pt,
               frame=lines,
               framesep=2mm]{python}
a = int(raw_input("Enter a:"))
b = int(raw_input("Enter b:"))
# Add here!
# r = ...
print "Remainder is", r
\end{minted}

\item The following code calculates the ceiling function $\lceil x \rceil = \min \{n \in \mathbb{Z} \mid n \geq x\}$ of a given positive real number \texttt{x}.
Use of \texttt{int($\cdot$)}, \texttt{+}, and \texttt{-} is sufficient.
Add additional code.
\begin{itemize}
\item Hint: Try plotting \texttt{int($\cdot$)} function.
\end{itemize}
\begin{minted}[mathescape,
               linenos,
               breaklines,
               numbersep=5pt,
               frame=lines,
               framesep=2mm]{python}
x = float(raw_input("Enter x:"))
# Add here!
# ceil_x = ...
print "Ceil(x) =", ceil_x
\end{minted}
\end{enumerate}
\end{document}