\documentclass[../main.tex]{subfiles}

\begin{document}

\subsection{Boolean Functions불리언 함수}
지난 시간에는 조건문을 배우면서 불리언 자료형에 대해 알아보았습니다.
그렇다면 불리언 자료형인 \texttt{True}와 \texttt{False}를 반환하는 Boolean functions불리언 함수도 생각해볼 수 있습니다.
이미 함수와 조건문을 배웠기 때문에, 이번 절의 내용은 매우 간단합니다.
지난 시간의 복습이라고 보아도 무방합니다.
이번 절에서는 불리언 함수를 정의하고 사용할 때 무엇을 주의해야 하는 지에 대해 간략히 알아봅니다.

\subsubsection{Boolean Functions불리언 함수의 예시}
불리언 함수는 불리언 자료형 \texttt{True}와 \texttt{False}를 반환하는 함수입니다.
불리언 함수는 길고 복잡한 조건문을 간략하게 표현할 때 유용합니다.
아래는 길이 \texttt{a}, \texttt{b}, \texttt{c}를 가지는 선분이 삼각형을 이룰 수 있는지 판별하는 조건문입니다.
\begin{minted}[mathescape,
               linenos,
               breaklines,
               numbersep=5pt,
               frame=lines,
               framesep=2mm]{python}
if a < b + c and b < c + a and c < a + b:
    ...
\end{minted} 
이는 아래와 같은 함수로 대신할 수 있습니다.
\begin{minted}[mathescape,
               linenos,
               breaklines,
               numbersep=5pt,
               frame=lines,
               framesep=2mm]{python}
def triangle_formed(a, b, c):
    if not a < b + c:  # if a >= b + c:
        return False
    if not b < c + a:
        return False
    if not c < a + b:
        return False
    return True

if triangle_formed(a, b, c):
    ...
\end{minted}

복잡한 조건문의 경우에는 불리언 함수의 유용성이 명백해집니다.
``두 수 $n_1$과 $n_2$가 서로소이라면'' 등의 표현을 한 줄로 표현하는 것보다는, 두 수가 서로소이면 \texttt{True}를 아니면 \texttt{False}를 반환하는 함수를 정의한 뒤--예컨대 \texttt{coprime(n1, n2)}--\texttt{if coprime(n1, n2):}와 같이 정리하는 것이 바람직합니다.
나아가, 이러한 불리언 함수는 조건 확인의 재사용성을 높여줍니다.
예를 들어, 로봇이 거리를 체크한 후 장애물이 앞에 있는지 확인하여 불리언을
반환하는 함수를 생각해봅시다.
\begin{minted}[mathescape,
               linenos,
               breaklines,
               numbersep=5pt,
               frame=lines,
               framesep=2mm]{python}
def check_obstacle(distance):
    if distance < 10:
        return True
    return False
\end{minted}
언뜻 보면 거리를 확인해야할 곳에 \verb|distance < 10|을 써넣는 것이 굳이 함수를
작성하는 것보다 간단해 보일 수도 있지만, 로봇을 조종하는 코드에서 장애물 확인은
자주 발생할 것입니다.
그런데 만약 장애물이 있다고 인식하는 거리를 10에서 5로 줄이고 싶다고 합시다.
만약 \verb|distance < 10|로 거리를 확인했다면, 코드의 수십, 아니 수백 곳에서
10을 5로 바꿔줘야 할 것입니다.
반면 함수를 정의한 후 \verb|check_obstacle(distance)|와 같이 사용했다면 단 한
곳만 수정하면 될 것입니다.

\subsubsection{유의 사항}
불리언 \texttt{True}는 문자열 \texttt{"True"}가 아닙니다.
마찬가지로 불리언 \texttt{False}는 문자열 \texttt{"False"}가 아닙니다.
\begin{minted}[mathescape,
               linenos,
               breaklines,
               numbersep=5pt,
               frame=lines,
               framesep=2mm]{python}
>>> type(True)
<type 'bool'>
>>> type("True")
<type 'str'>
>>> type(False)
<type 'bool'>
>>> type("False")
<type 'str'>
\end{minted}
\texttt{True}는 문자열이 아니라 그 자체로 불리언 자료형을 지닌 값입니다.\footnote{물론, 지난 시간에 언급했듯이 \texttt{True}는 1, \texttt{False}는 0입니다.}
이를 강조하는 이유는, 간혹 \verb|True|를 반환해야할 곳에 실수로
\verb|"True"|라고 적는 경우가 있기 때문입니다.
이 둘은 서로 다른 값입니다.

조건문은 그 자체로 불리언 자료형을 가지기 때문에, \texttt{if 조건문: return True(False)}와 같은 형태는 단순히 \texttt{return 조건문}으로 줄여 쓸 수 있습니다.
아래는 $\texttt{a}x^2 + \texttt{b}x + c = 0$의 실근이 존재하는지를 판별하는 함수 \texttt{real\_solution(a, b, c)}입니다.
\begin{minted}[mathescape,
               linenos,
               breaklines,
               numbersep=5pt,
               frame=lines,
               framesep=2mm]{python}
def real_solution(a, b, c):
    if b**2 - 4*a*c >= 0:
        return True
    return False
\end{minted}
이는 간략히 
\begin{minted}[mathescape,
               linenos,
               breaklines,
               numbersep=5pt,
               frame=lines,
               framesep=2mm]{python}
def real_solution(a, b, c):
    return b**2 - 4*a*c >= 0
\end{minted}
으로 나타낼 수 있습니다.
반대로 \texttt{if 조건문: return False}는 \texttt{return not 조건문}과 같이 정리가 됩니다.

나아가 이런 불리언 함수를 조건문에 사용할 때, \texttt{if 불리언\_함수 == True:} 대신 \texttt{if 불리언\_함수:}를 쓰는 것이 바람직합니다.
즉,
\begin{minted}[mathescape,
               linenos,
               breaklines,
               numbersep=5pt,
               frame=lines,
               framesep=2mm]{python}
if real_solution(a, b, c) == True:
    ...
\end{minted}
보다는
\begin{minted}[mathescape,
               linenos,
               breaklines,
               numbersep=5pt,
               frame=lines,
               framesep=2mm]{python}
if real_solution(a, b, c):
    ...
\end{minted}
이 더 간결한 표현입니다.
특히 \texttt{if 불리언\_함수 == True:}을 쓰려다가 \texttt{==} 대신 \texttt{=}을 사용하는 실수를 미연에 방지할 수 있습니다.

그리고 \emph{복잡한 표현은 끊어서 표현하는 것이 미연의 실수를 방지하고 코드의 가독성을 높입니다.}
다음 코드
\begin{minted}[mathescape,
               linenos,
               breaklines,
               numbersep=5pt,
               frame=lines,
               framesep=2mm]{python}
if ((x1 - x2)**2 + (y1 - y2)**2)**.5 <= ((x2 - x3)**2 + (y2 - y3)**2)**.5 + ((x3 - x1)**2 + (y3 - y1)**2)**.5:
    ...
\end{minted}
보다는
\begin{minted}[mathescape,
               linenos,
               breaklines,
               numbersep=5pt,
               frame=lines,
               framesep=2mm]{python}
a = ((x1 - x2)**2 + (y1 - y2)**2)**.5
b = ((x2 - x3)**2 + (y2 - y3)**2)**.5
c = ((x3 - x1)**2 + (y3 - y1)**2)**.5
if a < b + c:
    ...
\end{minted}
이 훨씬 읽기 편한 것을 볼 수 있습니다.

마지막으로, \textbf{\texttt{float}형 수끼리는 등호 비교 연산\texttt{==}을 하면 안 됩니다.}
\begin{minted}[mathescape,
               linenos,
               breaklines,
               numbersep=5pt,
               frame=lines,
               framesep=2mm]{python}
>>> 0.1 + 0.2
0.30000000000000004
>>> 0.1 + 0.2 == 0.3
False
\end{minted}
이러한 결과는 컴퓨터의 부동 소수점 처리 방식 때문인데, 부동 소수점 rounding error반올림 오차라고 합니다.
간단히는 이진수 소수로 십진수 소수를 정확하게 나타낼 수 없어서, 오차가 누적되어 생기는 오차라고 생각하시면 됩니다.\footnote{자세한 내용은 \url{https://en.wikipedia.org/wiki/IEEE_754}와 \url{https://docs.oracle.com/cd/E19957-01/806-3568/ncg_goldberg.html}에 자세히 나와 있습니다.}
따라서 \texttt{float}끼리의 비교는 \texttt{>}와 \texttt{<}만 신뢰할 수 있으며, 되도록이면 \texttt{int} 상태에서 비교를 하는 것이 좋습니다.
예를 들어 두 정수 격자점 사이의 거리를 비교할 때는 제곱근을 씌우지 않은 상태에서 비교를 해야 정확합니다.

\subsection{Loops반복문 기본}
컴퓨터는 인간보다 단순 반복 계산을 빠르게 수행할 수 있습니다.
이번 절에서 배울 loops반복문을 익히면 손으로는 할 수 없는 많은 계산을 Python으로 수행할 수 있을 것입니다.
나아가 \emph{지금까지 배운 함수, \texttt{if-else} 문과 같이 사용한다면, 원하는
작업을 하는 코드를 설계하고 작성할 수 있을 것입니다.}

간단한 예시로는 로봇에게 반복적인 작업을 수행하도록 반복문을 통해 작성할 수
있습니다.
로봇이 특정 거리만큼 앞으로 가게 하는 함수 \verb|move_forward(distance)|, 특정
각도만큼 우회전하는 함수 \verb|turn_right(angle)|가 주어져 있다고 합시다.
이때 정사각형으로 한 바퀴를 돌게 하는 코드는 Python의 반복문을 사용해 다음과
같이 쓸 수 있습니다.
\begin{minted}[mathescape,
               linenos,
               breaklines,
               numbersep=5pt,
               frame=lines,
               framesep=2mm]{python}
for i in range(4):
    move_forward(1)
    turn_right(90)
\end{minted}
줄 2와 줄 3의 내용을 총 4번 반복하도록 하는 \verb|for| 반복문으로, 어떤 일을
하는 코드인지 어렵지 않게 이해할 수 있습니다.

첫 번째 수업에서 1부터 $n$까지의 수를 더하는 문제를 풀 때에는 공식
$1 + 2 + \dots + n = \frac{n(n + 1)}{2}$을 사용하였습니다.
그렇다면 $1^m + 2^m + \dots + n^m$은 어떻게 계산할 수 있을까요?
임의의 $m$에 대해서 공식을 유도할 수는 없습니다.
이런 경우에 \texttt{for} 반복문을 활용할 수 있습니다.
지금까지는 `$n$회 무슨 작업을 반복하라'의 명령을 하려면 직접 $n$회 칠 수 밖에 없었지만,  \texttt{for} 반복문을 사용하면 임의의 $n$에 대해서 단 한 번의 일반적인 규칙을 제시하면 명령을 수행할 수 있습니다.
$\sum_{i = 1}^n i^m$을 반환하는 함수는 아래와 같습니다.
\begin{minted}[mathescape,
               linenos,
               breaklines,
               numbersep=5pt,
               frame=lines,
               framesep=2mm]{python}
def summ(m, n):
    summ = 0
    for i in range(1, n + 1):
        summ += i**m
    return summ
\end{minted}
줄 3에서 \texttt{i}를 1 이상 \texttt{n + 1} 미만까지 하나씩 증가하며 밑의 줄 4를 반복한다는 뜻입니다.
이처럼 \texttt{for} 반복문의 기본은 \texttt{range($\cdot$)}이라고 볼 수 있습니다.
이번 절에서는 \texttt{range($\cdot$)} 함수를 활용하여, 조건문을 활용한 반복문이나 nested loops중첩 반복문에 대해 자세히 알아봅니다.

\subsubsection{\texttt{range($\cdot$)} 함수를 활용한 \texttt{for} 반복문}
\texttt{range($\cdot$)} 함수\footnote{\texttt{range($\cdot$)}는 리스트를 반환하는 함수로, 리스트에 대해서는 다음 시간에 자세히 알아볼 것입니다.}는 인자를 한 개에서 세 개까지 넣을 수 있습니다.
일단 하나를 넣는 경우를 봅니다.
\begin{minted}[mathescape,
               linenos,
               breaklines,
               numbersep=5pt,
               frame=lines,
               framesep=2mm]{python}
for i in range(n):
    print(i)
\end{minted}
이는 
\begin{minted}[mathescape,
               linenos,
               breaklines,
               numbersep=5pt,
               frame=lines,
               framesep=2mm]{python}
0
1
2
...
n - 1
\end{minted}
를 출력합니다.
즉, \texttt{range(n)}은 0에서 \texttt{n}이 되기 직전까지 \texttt{i}를 1씩
증가시키며 \texttt{print(i)}을 실행합니다.
참고로, 여기서 함수 밖에서 \texttt{for} 반복문이 정의되었다면 \texttt{i}는 전역 변수이므로, 다시 정의하지 않으면 \texttt{n - 1}로 남아있게 됩니다.

\texttt{range($\cdot$)}에 두 개의 인자가 전달되는 경우가 아래에 나타나 있습니다.
\begin{minted}[mathescape,
               linenos,
               breaklines,
               numbersep=5pt,
               frame=lines,
               framesep=2mm]{python}
for i in range(m, n):
    print(i)
\end{minted}
이는 
\begin{minted}[mathescape,
               linenos,
               breaklines,
               numbersep=5pt,
               frame=lines,
               framesep=2mm]{python}
m
m + 1
m + 2
...
n - 1
\end{minted}
와 동일합니다.
\texttt{range(m, n)}은 \texttt{m}에서 \texttt{n}이 되기 직전까지 \texttt{i}를
1씩 증가시키며 \texttt{print(i)}를 실행하는데, $\texttt{m} \geq \texttt{n}$이라면 반복문이 수행되지 않습니다.

마지막으로 세 개의 인자가 \texttt{range($\cdot$)}에 전달되는 경우를 살펴봅니다.
\begin{minted}[mathescape,
               linenos,
               breaklines,
               numbersep=5pt,
               frame=lines,
               framesep=2mm]{python}
for i in range(m, n, k):
    print(i)
\end{minted}
이는
\begin{minted}[mathescape,
               linenos,
               breaklines,
               numbersep=5pt,
               frame=lines,
               framesep=2mm]{python}
m
m + k
m + 2*k
...
m + l*k
\end{minted}
와 같습니다.
위에서 \texttt{m + l*k}는 \texttt{n}이 되기 직전의 값입니다.
\texttt{range(m, n, k)}은 \texttt{m}에서 \texttt{n}이 되기 직전까지
\texttt{i}를 \texttt{k}씩 증가시키며 \texttt{print(i)}를 실행하는 것입니다.
\texttt{k}는 음수가 될 수도 있습니다.
이 경우에는 \texttt{i}가 \texttt{m}에서부터 \texttt{n}이 되기 직전까지 감소됩니다.

정리하자면 \texttt{range(n)}은 \texttt{range(0, n)}, \texttt{range(0, n, 1)}과 같고, \texttt{range(m, n)}은 \texttt{range(m, n, 1)}과 같습니다.
아래는 인자를 넣는 세가지 경우를 모두 나타낸 예시입니다.
\begin{minted}[mathescape,
               linenos,
               breaklines,
               numbersep=5pt,
               frame=lines,
               framesep=2mm]{python}
for i in range(n):
    print(i, end=' ')  # 0 1 ... n - 1

for i in range(1, n + 1):
    print(i, end=' ')  # 1 2 ... n

for i in range(10, 0, -1):
    print(i, end=' ')  # 10 9 ... 1

for i in range(1, 10, 2):
    print(i, end=' ')  # 1 3 ... 9
\end{minted}
위 예시에서 \verb|print()|안에 \verb|end=' '|를 써준 것은 줄바꿈을 하지 않고
대신 띄어쓰기를 하기 위함입니다.
Python의 내장 \verb|print()| 함수에서는 \verb|end|에 값을 넣어 값을 출력한 후
뒤에 붙일 문자를 직접 지정할 수 있습니다.

\subsubsection{예시}
이제는 \texttt{for} 문을 사용한 다양한 예시를 살펴보겠습니다.
아래는 팩토리얼을 계산하는 함수를 \texttt{for} 문을 통해 만든 것입니다.
\begin{minted}[mathescape,
               linenos,
               breaklines,
               numbersep=5pt,
               frame=lines,
               framesep=2mm]{python}
def factorial(x):
    product = 1
    for i in range(1, x + 1):
        product *= i
    return product
\end{minted}
줄 2에서 \texttt{product}의 값을 1로 초기화를 한 후, 이후 반복하여
\verb|product|에 \verb|i|를 1부터 \verb|x|까지 반복하며 곱해주었습니다.
\emph{\texttt{range}에서 마지막 인자는 해당 값이 되기 이전까지 반복문을
수행해줍니다.}

구구단표 작성과 같은 규칙적인 작업도 \texttt{for} 문을 통해 할 수 있습니다.
\begin{minted}[mathescape,
               linenos,
               breaklines,
               numbersep=5pt,
               frame=lines,
               framesep=2mm]{python}
for i in range(1, 10):
    for j in range(1, 10):
        print(i * j, end=' ')
    print()
\end{minted}
\texttt{i}를 고정시킨 후 \texttt{j}를 1부터 9까지 변화시키며 곱을 출력한 후, 줄을 바꾼 후 \texttt{i}를 1 증가시킨후 \texttt{j}를 1부터 9까지 변화시키며 곱을 출력하는 것을 반복하는 것이므로 결과적으로 구구단표를 작성하게 됩니다.
이렇게 \texttt{for} 문 안에서 또 다른 \texttt{for} 문이 수행되는 것을 중첩 반복문이라고 합니다.

코딩에서 개수 세기 등과 함께 중요한 패턴 중 하나는 최댓값과 최솟값을 찾는 것입니다.
$f(n) = (x - 2)^2$으로 정의된 함수에서 $f(0), \dots, f(4)$ 중 가장 작은 값을
찾고 싶다면 다음과 같이 코드를 작성할 수 있습니다.
\begin{minted}[mathescape,
               linenos,
               breaklines,
               numbersep=5pt,
               frame=lines,
               framesep=2mm]{python}
def f(x):
    return (x - 2)**2

def find_min_value():
    m = f(0)

    for i in range(1, 5):
        if f(i) < m:
            m = f(i)

    return m
\end{minted}
위 코드는 첫 번째 함숫값부터 마지막 함숫값까지 하나씩 살펴보는 것과 같습니다.
첫 번째 값을 보았을 때는 해당 값이 제일 작은 것(줄 5)이고 이후로는 이전까지의
최솟값보다 크면(줄 7) 해당 값으로 최솟값이 수정되는 것(줄 8)입니다.
최댓값을 구하는 것도 마찬가지로, 줄 8의 부등호 방향을 반대로 바꾸면 최댓값을 구하는 코드가 됩니다.
어떤 주어진 값에서 제일 작거나 큰 값을 찾을 때 자주 사용하는 패턴입니다.

% \subsection{예제}
% \begin{enumerate}
% \item Write a function \texttt{is\_odd(n)} that returns \texttt{True} if integer \texttt{n} is odd and \texttt{False} if integer \texttt{n} is even.
% One line of code is sufficient.
% \begin{minted}[mathescape,
%                linenos,
%                breaklines,
%                numbersep=5pt,
%                frame=lines,
%                framesep=2mm]{python}
% def is_odd(n):
%     # Add here!
%
% print type(is_odd(4))
% print is_odd(12)  # False
% print is_odd(1)   # True
% \end{minted}
% Write a function \texttt{exactly\_two\_even(n1, n2, n3)} by using \texttt{is\_odd($\cdot$)} above. \texttt{exactly\_two\_even(n1, n2, n3)} returns \texttt{True} if exactly two of \texttt{n1}, \texttt{n2}, and \texttt{n3} are even and \texttt{False} otherwise.
% \begin{minted}[mathescape,
%                linenos,
%                breaklines,
%                numbersep=5pt,
%                frame=lines,
%                framesep=2mm]{python}
% def exactly_two_even(n1, n2, n3):
%     cnt = 0
%     # Add here!
%     return cnt == 2
%
% print exactly_two_even(1, 2, 3)   # False
% print exactly_two_even(4, 2, 3)   # True
% print exactly_two_even(7, 3, 4)   # False
% print exactly_two_even(2, 5, 8)   # True
% print exactly_two_even(2, 18, 4)  # False
% \end{minted}
%
% \item Write a function \texttt{triangle(a, b, c)} that returns \texttt{True} if it is possible to form a triangle with sides of lengths \texttt{a}, \texttt{b}, and \texttt{c}.
% Assume \texttt{a}, \texttt{b}, and \texttt{c} are positive integers.
% \begin{minted}[mathescape,
%                linenos,
%                breaklines,
%                numbersep=5pt,
%                frame=lines,
%                framesep=2mm]{python}
% # Add here!
%
% print triangle(3, 4, 5)  # True
% print triangle(1, 5, 2)  # False
% print triangle(1, 1, 1)  # True
% print triangle(3, 1, 1)  # False
% \end{minted}
%
% \item Write a function \texttt{acute(x1, y1, x2, y2, x3, y3)} that returns \texttt{True} if three points \texttt{(x1, y1)}, \texttt{(x2, y2)}, and \texttt{(x3, y3)} form an acute-angled triangle(예각 삼각형) and \texttt{False} otherwise.
% \begin{minted}[mathescape,
%                linenos,
%                breaklines,
%                numbersep=5pt,
%                frame=lines,
%                framesep=2mm]{python}
% # Add here!
%
% print acute(1, 2, 4, 3, 2, 7)  # True
% print acute(1, 2, 4, 2, 5, 4)  # False
% print acute(1, 2, 4, 2, 4, 3)  # False
% \end{minted}
%
% \item Two circles are represented by centers \texttt{(x1, y1)} and \texttt{(x2, y2)} with radii \texttt{r1} and \texttt{r2}.
% Write a function \texttt{intersect(x1, y1, r1, x2, y2, r2)} that returns \texttt{True} if the two circles intersect at two points and \texttt{False} otherwise.
% Use \texttt{abs($\cdot$)}.
% \begin{minted}[mathescape,
%                linenos,
%                breaklines,
%                numbersep=5pt,
%                frame=lines,
%                framesep=2mm]{python}
% # Add here!
%
% print intersect(1, 1, 3, 5, 4, 2)  # False
% print intersect(1, 1, 3, 4, 3, 2)  # True
% print intersect(1, 1, 3, 2, 1, 2)  # False
% \end{minted}
%
% \item If a year is divisible by four, it is a leap year(윤년).
% However, if the year is divisible by 100 but not by 400, it is not a leap year.
% Write a function \texttt{leap\_year(year)} that returns \texttt{True} if \texttt{year} is a leap year and \texttt{False} otherwise.
% \begin{minted}[mathescape,
%                linenos,
%                breaklines,
%                numbersep=5pt,
%                frame=lines,
%                framesep=2mm]{python}
% # Add here!
%
% print leap_year(2008), leap_year(2011)  # True False
% print leap_year(2000), leap_year(2100)  # True False
% print leap_year(2300), leap_year(2400)  # False True
% print leap_year(2012), leap_year(2200)  # True False
% \end{minted}
% Write a function \texttt{num\_days(year, month)} that returns the number of days in month \texttt{month} of year \texttt{year}, using \texttt{leap\_year($\cdot$)}.
% \texttt{month} is given by an integer.
% \begin{minted}[mathescape,
%                linenos,
%                breaklines,
%                numbersep=5pt,
%                frame=lines,
%                framesep=2mm]{python}
% def num_days(year, month):
%     assert 1 <= month <= 12
%     if month == 1 or month == 3 or month == 5 or month == 7 or \
%        month == 8 or month == 10 or month == 12:
%         return 31
%     # Add here!
%
% print num_days(2000,1), num_days(2001,4), num_days(2004,8)  # 31 30 31
% print num_days(2004,9), num_days(2005,3), num_days(2005,7)  # 30 31 31
% print num_days(2008,2), num_days(2011,2), num_days(2012,2)  # 29 28 29
% print num_days(2000,2), num_days(2100,2), num_days(2200,2)  # 29 28 28
% print num_days(2300,2), num_days(2400,2), num_days(3200,2)  # 28 29 29
% \end{minted}
%
% \item Blood transfusion can be done as:
% \begin{itemize}
% \item Type O from O
% \item Type A from A or O
% \item Type B from B or O
% \item Type AB from A, B, AB, or O.
% \end{itemize}
% Write a function \texttt{blood(supply\_O, supply\_A, supply\_B, supply\_AB, demand\_O, demand\_A, demand\_B, demand\_AB)} that returns \texttt{True} if the current supply of blood suffice the current demand and \texttt{False} otherwise.
% \begin{minted}[mathescape,
%                linenos,
%                breaklines,
%                numbersep=5pt,
%                frame=lines,
%                framesep=2mm]{python}
% def blood (supply_O, supply_A, supply_B, supply_AB,
%            demand_O, demand_A, demand_B, demand_AB):
%     if supply_O < demand_O:
%         return False
%     # Add here!
%
% print blood(50, 36, 11, 8, 45, 42, 10, 3)  # False
% print blood(50, 36, 11, 3, 45, 38, 10, 7)  # True
% \end{minted}
%
% \item The following output is desired:
% \begin{minted}[mathescape,
%                linenos,
%                breaklines,
%                numbersep=5pt,
%                frame=lines,
%                framesep=2mm]{python}
% 1 2 3 4 5 6 7 8 9
% 3 6 9 12 15 18 21 24 27
% 5 10 15 20 25 30 35 40 45
% 7 14 21 28 35 42 49 56 63
% 9 18 27 36 45 54 63 72 81
%
% 1
% 3 6 9
% 5 10 15 20 25
% 7 14 21 28 35 42 49
% 9 18 27 36 45 54 63 72 81
% \end{minted}
% Write a function print\_tables() that prints the above.
% \begin{minted}[mathescape,
%                linenos,
%                breaklines,
%                numbersep=5pt,
%                frame=lines,
%                framesep=2mm]{python}
% def print_tables():
%     # Add here!
%
% print_tables()
% \end{minted}
%
% \item Write a function \texttt{sum\_interval(a, b)} that returns $\texttt{a} + (\texttt{a} + 1) + \dots + \texttt{b}$ where \texttt{a} and \texttt{b} are integers.
% \begin{minted}[mathescape,
%                linenos,
%                breaklines,
%                numbersep=5pt,
%                frame=lines,
%                framesep=2mm]{python}
% def sum_interval(a, b):
%     # Add here!
%
% print sum_interval(5, 10)
% print sum_interval(15, 100)
% \end{minted}
%
% \item The following relation is known\footnote{Gottfried Wilhelm Leibniz이 발견하였다.}:
% $$
% \pi = 4\left(\frac11 - \frac13 + \frac 15 - \frac17 + \frac19 - \dots\right)
% $$
% Write a function \texttt{calculate\_pi(n)} that returns an approximate value of $\pi$ using the first \texttt{n} terms of the above relation.
% \begin{minted}[mathescape,
%                linenos,
%                breaklines,
%                numbersep=5pt,
%                frame=lines,
%                framesep=2mm]{python}
% def calculate_pi(n):
%     # Add here!
%
% print calculate_pi(1000)
% \end{minted}
%
% \item Write a function \texttt{within\_circ(r)} that returns the number of integer lattice points within a circle of radius \texttt{r} centered at the origin.
% The following relation results from an Monte Carlo method:
% $$
% \pi = \lim_{\texttt{r} \rightarrow \infty} \frac{\texttt{within\_circle(r)}}{\texttt{r}^2}
% $$
% Write a function \texttt{calculate\_pi(r)} that approximates $\pi$ using the above equation.
% \begin{minted}[mathescape,
%                linenos,
%                breaklines,
%                numbersep=5pt,
%                frame=lines,
%                framesep=2mm]{python}
% def within_circ(r):
%     cnt = 0
%     # Add here!
%     return cnt
%
% def calculate_pi(r):
%     # Add here!
%
% print calculate_pi(1000)
% \end{minted}
%
% \item Write a function \texttt{day\_of\_week(year, month, day)} that returns the day of the week as a string \texttt{"Mon"}, \texttt{"Tue"}, \texttt{"Wed"}, \texttt{"Thu"}, \texttt{"Fri"}, \texttt{"Sat"}, and \texttt{"Sun"}.
% Assume $\texttt{year} \geq 2000$ and \texttt{month} is given as an integer.
% Note that January 1 of 2000 is Saturday, and use \texttt{leap\_year($\cdot$)} and \texttt{num\_days($\cdot$)} from problem 5.
% \begin{minted}[mathescape,
%                linenos,
%                breaklines,
%                numbersep=5pt,
%                frame=lines,
%                framesep=2mm]{python}
% def day_of_week(year, month, day):
%     # Add here!
%
% print day_of_week(2018, 1, 20)  # Sat
% \end{minted}
%
% \item Let $f: \{0, 1, \dots, 99\}^2 \rightarrow \mathbb{Z}$ be a function defined by:
% $$
% f(i, j) = \left(i^5 + 2i^3j^2 + 5i^2 + j + 5000\right) \text{ mod $10000$}
% $$
% Write a function \texttt{find\_max()} that returns the maximum value of the function $f$.
% \begin{minted}[mathescape,
%                linenos,
%                breaklines,
%                numbersep=5pt,
%                frame=lines,
%                framesep=2mm]{python}
% def f(i, j):
%     return (i**5 + 2* i**3 * j**2 + 5 * i**2 + j + 5000) % 10000
%
% def find_max():
%     # Add here!
%
% print "Maximum is", find_max()  # 9997
% \end{minted}
%
% \item Write a function \texttt{sum\_mlt35(n)} that returns the sum of all multiples of 3 or 5 smaller than \texttt{n}.
% \begin{minted}[mathescape,
%                linenos,
%                breaklines,
%                numbersep=5pt,
%                frame=lines,
%                framesep=2mm]{python}
% def sum_mlt35(n):
%     # Add here!
%
% print sum_mlt35(10)    # 23
% print sum_mlt35(100)   # 2318
% print sum_mlt35(1000)  # 233168
% \end{minted}
%
% \item Write a function \texttt{sum\_even\_fib(n)} that returns the sum of the even-valued terms of the Fibonacci sequence in the first \texttt{n} terms, where the first and the second terms are 1 and 2, respectively.
% The calculation time should not exceed a second.
%
% \begin{minted}[mathescape,
%                linenos,
%                breaklines,
%                numbersep=5pt,
%                frame=lines,
%                framesep=2mm]{python}
% def sum_even_fib(n):
%     # Add here!
%
% print sum_even_fib(300) == ...   # Too large to write here
% print sum_even_fib(400) == ...   # Too large to write here
% print sum_even_fib(1000) == ...  # Too large to write here
% \end{minted}
%
% \item Write a function \texttt{d(n)} that returns the sum of proper divisors(진약수) of \texttt{n}.
% If $b = \texttt{d($a$)} = \texttt{d(d($b$))}$ where $a \neq b$, $a$ and $b$ are an amicable pair and each of $a$ and $b$ are called amicable(우호적인) numbers.
% Write a function \texttt{is\_amicable(n)} that returns \texttt{True} if \texttt{n} is amicable and \texttt{False} otherwise.
% Then write a function \texttt{sum\_amicable(n)} that returns the sum of all amicable numbers under \texttt{n}.
% Using \texttt{sum\_amicable(1000)}, evaluate the sum of all amicable numbers under 10000.
% It might take a while to execute; perhaps not if you find a better algorithm.
% \begin{minted}[mathescape,
%                linenos,
%                breaklines,
%                numbersep=5pt,
%                frame=lines,
%                framesep=2mm]{python}
% def d(n):
%     # Add here!
%
% def is_amicable(n):
%     # Add here!
%
% def sum_amicable(n):
%     # Add here!
%
% print sum_amicable(10000)  # 31626
% \end{minted}
% \end{enumerate}
\end{document}
