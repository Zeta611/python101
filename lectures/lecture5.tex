\documentclass[../main.tex]{subfiles}

\begin{document}
\section{한정자}
지금까지 개수 세기, 최소/최대값 구하기 등의 기본적인 반복문 패턴을 알아보았습니다.
이번에는 \alt*{한정자}{quantifier}[quantifiers] 패턴을 알아봅니다.

수학에서 사용하는 \alt*{전체 한정자}{universal quantifier} $\forall$와 \alt*{존재 한정자}{existential quantifier} $\exists$를 프로그램으로 작성한 것을 한정자 패턴이라고 합니다.
한정자 패턴을 통해 어떤 집합 $S$가 주어졌을 때,
\begin{itemize}
\item $\forall x \in S.\ p(x)$
\item $\exists x \in S.\ p(x)$
\end{itemize}
의 참/거짓을 판별할 수 있습니다.

또한 지난 단원에서 \texttt{hubo.cs1robots} 패키지를 통해 휴보를 원하는대로 움직이도록 코드를 작성했습니다.
이번에는 휴보가 줍고 놓을 수 있는 \alt*{비퍼}{beeper}[beepers]를 다룰 것입니다.
비퍼는 격자마다 배치할 수 있는데, 휴보는 해당 격자로 이동해야만 비퍼를 세거나 놓고 주울 수 있습니다.
그림~\ref{fig:lecture5beeper}처럼 생긴 지형에서 놓고 주울 수 있는 \pyin{drop_beeper()}와 \pyin{pick_beeper()} 메소드 등에 대해서 알아봅니다.

\begin{figure}[htbp]
\centering
\includegraphics[width=0.5\linewidth]{"./lectures/lecture5_beeper"}
\caption{비퍼가 놓인 지형입니다.}\label{fig:lecture5beeper}
\end{figure}

\begin{table}[hbtp]
  \caption{한정자가 사용된 예시입니다.}\label{tab:lecture5:quantifiers}
  \begin{tblr}{ccc}
    \toprule
        \texttt{numbers} & $\forall \texttt{i}.\ \texttt{numbers[i]}>0$ & $\exists \texttt{i}.\ \texttt{numbers[i]}>0$\\ \midrule
        \pyin{[1, 3, 2, 5, 2]} & \pyin{True} & \pyin{True}\\
        \pyin{[-1, -3, -2, -5, -2]} & \pyin{False} & \pyin{False} \\
        \pyin{[-1, -3, -2, 5, 2]} & \pyin{False} & \pyin{True}\\
        \bottomrule
    \end{tblr}
\end{table}
표~\ref{tab:lecture5:quantifiers}에서 리스트\texttt{numbers}의 원소가 모두 양인 경우, 일부만 양인 경우, 그리고 양의 원소가 없는 경우에 대해, 한정자가 적용된 식의 값들이 정리되어  있습니다.
전체 한정자의 경우 \texttt{numbers}의 모든 원소가 양수일 경우에만 \pyin{True} 값을 가지게 하며, 음의 원소가 하나라도 있을 경우에는 \pyin{False} 값을 가지게 합니다.
반면 존재 한정자의 경우 \texttt{numbers}의 원소 가운데 양인 것이 하나라도 있다면 \pyin{True}입니다.

한정자를 다루는 것은 곧 명제의 진리값을 구하는 것이기 때문에 불리언 함수로 계산할 수 있습니다.
또한 리스트의 각 원소를 확인해야 하므로 \texttt{for} 반복문을 사용해야 합니다.
전체 한정자의 경우 모든 원소가 어떤 조건을 만족하는지를 확인해야 합니다.
이는 거꾸로 조건을 만족시키지 못하는 단 하나의 원소라도 있다면 \pyin{False}를 반환하는 불리언 함수를 만들면 됩니다.
$\texttt{n} = \texttt{len(numbers)}$이면
\[
\forall \texttt{i} \in \texttt{range(n)}.\ p(\texttt{numbers[i]}) \Leftrightarrow \neg\left(\exists \texttt{i} \in \texttt{range(n)}.\ \neg p(\texttt{numbers[i]})\right)
\]
이기 때문입니다.
즉, 모든 원소를 조건문으로 확인하는 \texttt{for} 문에서 조건을 만족하지 못하면 바로 \pyin{False}를 반환하고, 루프를 통과하면 \pyin{True}를 반환하면 됩니다.

아래는 모든 원소가 양수인지를 확인하는 함수 \pyin{all_positive}입니다.
\begin{minted}{python}
def all_positive(numbers):
    for i in range(len(numbers)):
        if not numbers[i] > 0:
            return False
    return True
\end{minted}

소수를 판별하는 함수의 구현도 전체 한정자를 사용한 경우입니다.
1과 자기 자신이 아닌 약수가 단 하나라도 존재한다면 \pyin{False}를 반환하는 경우이기 때문입니다.
산술 기하 평균에 의해 해당 수의 절반까지의 루프를 구성했는데, 제곱근까지의 범위를 구성하여도 됩니다.

존재 한정자의 경우 조건을 만족시키는 단 하나의 원소라도 찾으면 됩니다.
이에 따라, 모든 원소를 조건문으로 확인하는 \texttt{for} 문에서 조건을 만족하는 원소를 발견하는 즉시 \pyin{True}를 반환하고, 반복문을 통과하면 \pyin{False}를 반환하면 됩니다.

아래는 하나의 원소라도 양수인지를 확인하는 함수 \pyin{some_positive}입니다.
\begin{minted}{python}
def some_positive(numbers):
    for i in range(len(numbers)):
        if numbers[i] > 0:
            return True
    return False
\end{minted}

\subsection{예시}
다음은 어떤 정수 \texttt{n}이 두 정수의 제곱의 합인지의 여부를 알려주는 함수 \texttt{sum\_squares}입니다.
\begin{minted}{python}
def sum_squares(n):
    sqrt_n = int(n ** .5)
    for x in range(1, sqrt_n + 1):
        for y in range(x, sqrt_n + 1):
            if n == x ** 2 + y ** 2:
                return True
    return False
\end{minted}
만약 $\texttt{x}^2 + \texttt{y}^2 = n^2$이라면 $\texttt{x}, \texttt{y} \leq \sqrt n$일 것이므로 \pyin{sqrt_n = int(n ** .5)}로 정의하여 줄 3, 4에서 \texttt{for} 반복문의 범위를 제한하였습니다.
나아가 줄 4에서는 $\texttt{x} \leq \texttt{y}$의 조건을 주어 계산할 값들의 범위를 줄였습니다.

아래는 주어진 리스트 \texttt{numbers}의 모든 원소가 두 정수의 제곱의 합인지를 판단하는 함수 \texttt{all\_sum\_squares}입니다.
위에서 구한 불리언 함수 \texttt{sum\_squares}를 이용한 전체 한정자 패턴입니다.
\begin{minted}{python}
def all_sum_squares(numbers):
    for i in range(len(numbers)):
        if not sum_squares(numbers[i]):
            return False
    return True
\end{minted}

마지막으로 주어진 수열이 증가 수열인지를 확인하는 함수 \texttt{inc\_seq}입니다.
\begin{minted}{python}
def inc_seq(numbers):
    for i in range(len(numbers) - 1):
        if not numbers[i] <= numbers[i + 1]:
            return False
    return True
\end{minted}
전체 한정자 패턴과 동일하지만, 주의할 점은 줄 2에서 \pyin{range}의 범위를 \pyin{len(numbers) - 1}로 해야 \texttt{IndexError} 예외가 나지 않습니다.
이는 줄 3에서 다음 인덱스의 원소와 현재 인덱스의 원소를 비교하기 때문입니다.

지금까지 배운 \texttt{for} 반복문의 패턴 세 가지를 아래에 정리합니다.
\begin{enumerate}
\item 최대/최소 구하기
\begin{minted}{python}
def maximum(numbers):  # def minimum(numbers):
    m = numbers[0]
    for i in range(len(numbers)):
        if numbers[i] > m:  # if numbers[i] < m:
            m = numbers[i]
    return m
\end{minted}
\item 카운터 패턴
\begin{minted}{python}
def cnt_odd(numbers):
    cnt = 0
    for i in range(len(numbers)):
        if numbers[i] % 2:
            cnt += 1
    return cnt
\end{minted}
\item 한정자 패턴
\begin{minted}{python}
def all_odd(numbers):  # def some_odd(numbers):
    for i in range(len(numbers)):
        if not numbers[i] % 2:  # if numbers[i] % 2:
            return False  # return True
    return True  # return False
\end{minted}
\end{enumerate}

\subsection{비퍼}
비퍼는 휴보가 놓고 주울 수 있는 물체로, 격자마다 위치할 수 있습니다.
휴보가 비퍼를 놓기 위해서는 \pyin{drop_beeper()} 메소드를 사용하면 됩니다.
이때 휴보는 \pyin{on_beeper()} 메소드로 비퍼를 감지할 수 있습니다.
만약 격자점에 비퍼가 놓여 있었다면, \pyin{pick_beeper()}로 주우면 됩니다.
이렇게 주운 비퍼는 휴보가 가지고 있다가 다른 격자에 다시 놓을 수 있습니다.
만약 비퍼가 놓이지 않은 격자에서 비퍼를 주우려 하면 에러가 발생합니다.
이에 따라 줍기 전에 항상 \pyin{on_beeper()}를 통해 현재 휴보가 위치한 격자에 비퍼가 있는지 확인을 해야 합니다.
아래는 \pyin{on_beeper()}를 통해 비퍼가 있는지 확인한 후, 있다면 \pyin{pick_beeper()}를 통해 비퍼를 줍는 코드입니다.
그림~\ref{fig:lecture5pickbeeper}에 실행 전후 결과가 나타나 있습니다.
\begin{minted}{python}
for i in range(9):
    hubo.move()
    if hubo.on_beeper():
        hubo.pick_beeper()
      \end{minted}

\begin{figure}[htbp]
\centering
\begin{subfigure}{.5\textwidth}
\centering
\includegraphics[width=.9\linewidth]{"./lectures/lecture5_pickbeeperbef"}
\end{subfigure}%
\begin{subfigure}{.5\textwidth}
\centering
\includegraphics[width=.9\linewidth]{"./lectures/lecture5_pickbeeperaft"}
\end{subfigure}
\caption{휴보가 비퍼를 반복해서 줍는 모습입니다.}\label{fig:lecture5pickbeeper}
\end{figure}

휴보가 처음에 가지고 있는 비퍼의 개수를 설정할 수도 있습니다.
지금까지는 단순히 \texttt{Robot()}으로 휴보를 생성하였지만, \pyin{Robot(beepers = 1)}과 같이 인자를 전달해주어 초기에 가지고 있는 비퍼의 개수를 지정할 수 있습니다.

\pyin{drop_beeper()}를 통해 비퍼를 가지고 있는 것보다 많이 놓으려 하면 에러가 발생하기 때문에, \pyin{carries_beeper()}로 비퍼를 가지고 있는지 확인을 한 후 놓을 수 있습니다.
\pyin{carries_beeper()} 메소드는 휴보가 현재 비퍼를 가지고 있는지 여부를 확인하는 불리언 함수입니다.
아래는 \texttt{beepers = 6} 인자를 통해 휴보가 처음에 여섯 개의 비퍼를 가지고 시작하도록 한 뒤, \pyin{carries_beeper()}를 통해 비퍼가 남아 있다면 하나씩 떨어뜨리면서 전진하도록 하는 코드입니다.
그림~\ref{fig:lecture5putbeeper}에 실행 전후 모습이 나타나 있습니다.
\begin{minted}{python}
hubo = Robot(beepers = 6)

for i in range(9):
    hubo.move()
    if hubo.carries_beeper():
        hubo.drop_beeper()
      \end{minted}

\begin{figure}[htbp]
\centering
\begin{subfigure}{.5\textwidth}
\centering
\includegraphics[width=.9\linewidth]{"./lectures/lecture5_pickbeeperbef"}
\end{subfigure}%
\begin{subfigure}{.5\textwidth}
\centering
\includegraphics[width=.9\linewidth]{"./lectures/lecture5_putbeeperaft"}
\end{subfigure}
\caption{휴보가 비퍼를 하나씩 놓으면서 이동하는 모습입니다.}\label{fig:lecture5putbeeper}
\end{figure}

\section{While문}
그런데 어떤 격자에 있는 비퍼를 모두 주우려면 어떻게 해야 할까요?
반복적인 행동을 취해야 하므로 \texttt{for}문을 활용해야 할 것 같은데, 매 격자점마다 몇 번 반복할지 정해지지 않았기 때문입니다.
이런 경우에 사용하는 것이 바로 \texttt{while}문입니다.

\texttt{while}문은 \texttt{for}문과 같은 반복문의 한 종류입니다.
\texttt{for}문은 명시적으로 몇 회 반복해야 하는지가 주어지지만, \texttt{while}문은 어떤 조건을 만족한다면 계속 명령을 반복하도록 하는 반복문입니다.

% 유클리드 호제법 같은 알고리즘을 구현하기 위해서는 \texttt{while}문을 활용해야 합니다:
% \begin{minted}{python}
% def gcd(m, n):
%     while n > 0:
%         m, n = n, m % n
%     return m
% \end{minted}
% \texttt{n}이 0보다 클 때, 즉 0이 되기 직전까지 줄 3의 내용을 실행하는 것입니다.
% 결국 \texttt{n}이 0이 되었을 때 \texttt{m}이 최대 공약수일 것이므로 \texttt{m}을 반환하는 코드입니다.

\subsection{While문과 휴보}
아래는 지난 절에서 beeper를 줍는 \texttt{pick\_beeper()} 메소드를 소개할 때 사용하였던 코드입니다.
\begin{minted}{python}
for i in range(9):
    hubo.move()
    if hubo.on_beeper():
        hubo.pick_beeper()
\end{minted}
그림~\ref{fig:lecture5pickbeeper}에 실행 결과가 담겨 있었습니다.

각 격자점에 하나 이상의 비퍼가 놓여있는 경우에는, 아래와 같이 \texttt{if}를 \texttt{while}로 바꾸면 됩니다.
\begin{minted}{python}
for i in range(9):
    hubo.move()
    while hubo.on_beeper():
        hubo.pick_beeper()
      \end{minted}
그림~\ref{fig:lecture5pickbeepers}에 결과가 나타나 있습니다.
\begin{figure}[htbp]
\centering
\begin{subfigure}{.5\textwidth}
\centering
\includegraphics[width=.9\linewidth]{"./lectures/lecture5_whilepickbeeper"}
\end{subfigure}%
\begin{subfigure}{.5\textwidth}
\centering
\includegraphics[width=.9\linewidth]{"./lectures/lecture5_pickbeeperaft"}
\end{subfigure}
\caption{비퍼가 한 격자에 여러 개 있는 경우 휴보가 비퍼를 반복해서 줍는 모습입니다.}\label{fig:lecture5pickbeepers}
\end{figure}
즉, \texttt{while}문은 \texttt{if}문을 여러 번 반복한 것과 동일한 것을 알 수 있습니다.

마찬가지로 비퍼를 놓을 때도 놓을 수 있는 만큼 최대한 놓으라는 명령을 \texttt{while}문으로 작성할 수 있습니다.
\begin{minted}{python}
hubo = Robot(beepers = 50)

while hubo.carries_beepers():
    hubo.drop_beeper()
hubo.move()
\end{minted}

\texttt{if}문에서 조건이 \pyin{False}이면 아예 실행이 되지 않는 것처럼 \texttt{while}문의 조건도 \pyin{False}로 시작될 경우 반복문이 실행되지 않고 넘어가게 됩니다.
따라서 휴보가 비퍼를 가지고 있지 않은 경우에는 \texttt{drop\_beeper()}가 실행되지 않습니다.

\texttt{while} 문을 활용해서 비퍼가 있는 곳까지 이동한 후 모두 주워서 돌아오라는 코드는 지금까지 배운 반복문을 활용하여 아래와 같이 작성할 수 있습니다.
\begin{minted}{python}
dist = 0

while not hubo.on_beeper():
    hubo.move()
    dist += 1

while hubo.on_beeper():
    hubo.pick_beeper()

hubo.turn_right()
hubo.turn_right()

for i in range(dist):
    hubo.move()
\end{minted}
줄 1에서 4까지는 \texttt{while} 문에서 카운터 패턴이 적용된 것을 볼 수 있습니다.
이는 나중에 휴보가 돌아올 때 몇 칸을 \texttt{move()}해야 할지 세야 하기 때문입니다.
이렇게 센 \texttt{dist} 값은 줄 13의 \texttt{for} 문에서 사용되었습니다.
줄 3의 \texttt{while} 문 조건의 경우, 휴보가 비퍼 위에 있지 않다면 앞으로 계속 이동하라는 줄 4의 명령을 수행하게 합니다.
그리고 해당 위치까지 이동해서 반복문을 탈출했을 때, 줄 7의 반복문이 수행되어 비퍼가 더 이상 없을 때까지 줍게 됩니다(줄 8).
마지막으로는 줄 13에서 이동한 칸 수만큼 \texttt{move()}하여 되돌아옵니다.

\section{예제}
\begin{enumerate}
\item 정수 \verb/n/이 세 양의 정수의 제곱의 합으로 나타내어질 수 있을 때 \pyin{True}를 반환하는 함수 \texttt{sum\_three\_squares(n)}를 작성하세요.
  예를 들어, 38은 $2^2 + 3^2 + 5^2$이므로 \pyin{True}를 반환해야 합니다.
\begin{minted}{python}
def sum_three_squares(n):
    m = int(n ** .5)
    # Add here!

for n in range(20, 31):
    print(n, sum_three_squares(n))

"""
20 False
21 True
22 True
23 False
24 True
25 False
26 True
27 True
28 False
29 True
30 True
"""
\end{minted}

\item 정수 \verb/n/이 \emph{서로 다른} 세 양의 정수의 제곱의 합으로 나타내어질 수 있을 때 \pyin{True}를 반환하는 함수 \texttt{sum\_three\_squares(n)}를 작성하세요.
\begin{minted}{python}
def sum_three_squares(n):
    # Add here!

for n in range(20, 31):
    print n, sum_three_dist_squares(n)

"""
20 False
21 True
22 False
23 False
24 False
25 False
26 True
27 False
28 False
29 True
30 True
"""
\end{minted}

\item 정수 \verb/n/이 두 소수의 합으로 나타내어질 수 있을 때 \pyin{True}를 반환하는 함수 \texttt{sum\_two\_primes(n)}를 작성하세요.
\begin{minted}{python}
def sum_two_primes(n):
    # Add here!

for n in range(20, 31):
    print(n, sum_two_primes(n))

"""
20 True
21 True
22 True
23 False
24 True
25 True
26 True
27 False
28 True
29 False
30 True
"""
\end{minted}

\item 정수 \verb/n/이 두 소수의 제곱의 합으로 나타내어질 수 있을 때 \pyin{True}를 반환하는 함수 \texttt{sum\_two\_prime\_squares(n)}를 작성하세요.
\begin{minted}{python}
def sum_two_prime_squares(n):
    # Add here!

for n in range(50, 61):
    print(n, sum_two_prime_squares(n))

"""
50 True
51 False
52 False
53 True
54 False
55 False
56 False
57 False
58 True
59 False
60 False
"""
\end{minted}

\item \verb/numbers/에 소수가 포함되어 있을 때만 \pyin{True}를 반환하는 함수 \texttt{some\_prime(numbers)}를 작성하세요.
  또한 \verb/numbers/의 모든 수가 소수일 때만 \pyin{True}를 반환하는 함수 \texttt{all\_primes(numbers)}를 작성하세요.
\begin{minted}{python}
def some_prime(numbers):
    # Add here!

def all_prime(numbers):
    # Add here!

num1 = [217, 287, 143, 163, 319]
num2 = [217, 287, 143, 169, 319]
num3 = [223, 281, 227, 151, 149]
print(some_prime(num1), all_prime(num1))  # True False
print(some_prime(num2), all_prime(num2))  # False False
print(some_prime(num3), all_prime(num3))  # True True
\end{minted}

\item \verb/numbers/에 속한 수들이 모두 다를 때에만 \pyin{True}를 반환하는 함수 \texttt{all\_distinct(numbers)}를 작성하세요.
\begin{minted}{python}
def all_distinct(numbers):
    # Add here!

print(all_distinct([1, 3, 2, 5, 2, 1]))  # False
print(all_distinct([1, 0, 2, 5, 3, 4]))  # True
\end{minted}

\item \verb/numbers/에 속한 모든 수들이 $\texttt{lower} \leq n \leq \texttt{upper}$를 만족할 때 \pyin{True}를 반환하는 함수 \texttt{all\_within\_range(numbers, lower, upper)}를 작성하세요.
\begin{minted}{python}
def all_within_range(numbers, lower, upper):
    # Add here!

print(all_within_range([1, 0, 2, 6, 3, 4], 0, 5))  # False
print(all_within_range([1, 0, 2, 5, 3, 4], 0, 5))  # True
\end{minted}

\item 위 두 문제에서 작성한 \verb/all_distinct/와 \verb/all_within_range/를 사용해서 \verb/numbers/가 \alt*{순열}{permutation}일 때 \pyin{True}를 반환하는 함수 \texttt{is\_permutation(numbers)}를 작성하세요.
정수 수열 $a_0, a_1, \dots, a_{n - 1}$에 속한 모든 수가 다르고 $i = 0, 1, \dots, n - 1$에 대해 $0 \leq n_i \leq n - 1$이면 수열 $a_i$는 순열입니다..
\begin{minted}{python}
def is_permutation(numbers, lower, upper):
    # Add here!

print(is_permutation([1, 3, 2, 5, 2, 1]))  # False
print(is_permutation([1, 0, 2, 5, 3, 4]))  # True
print(is_permutation([1, 0, 2, 6, 3, 4]))  # False
\end{minted}

\item 십진수로 나타낸 정수 \verb/n/의 표현에서 7을 세는 함수 \texttt{count\_sevens(n)}를 작성하세요.
\begin{minted}{python}
def count_sevens(n):
    # Add here!

print(count_sevens(1723))  # 1
print(count_sevens(1357924770))  # 3
\end{minted}

\item 두 양의 정수 \verb/a/와 \verb/b/의 최대 공약수를 계산하는 \texttt{gcd(a, b)}를 작성하세요.
\begin{minted}{python}
def gcd(a, b):
    if a < b:
        a, b = b, a

    while b > 0:
        # Add here!

    return # Add here!

print(gcd(36, 20))  # 4
print(gcd(2408208, 2790876))  # 132
\end{minted}
\end{enumerate}
\end{document}
