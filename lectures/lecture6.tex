\documentclass[../main.tex]{subfiles}

\begin{document}
지금까지 반복문의 다양한 패턴을 알아보았으니, 이를 정리하고 \texttt{for}문을 구성하는 새로운 방법과 \pyin{break}, \pyin{continue}으로 반복문을 직접 제어하는 방법에 대해서 알아볼 것입니다.

또한 텍스트 파일을 읽고 출력하는 파일 입출력에 대해서 간단히 살펴볼 것이며, 이를 통해 많은 양의 데이터를 처리할 수 있는 길이 열리게 됩니다.

\section{반복문의 응용}
\subsection{리스트 원소의 직접 접근}
지금까지는 리스트 \texttt{l}이 주어졌을 때 \pyin{range(len(l))}을 사용하여 다음과 같이 \texttt{for}문을 구성하였습니다.
\begin{minted}{python}
l = [2, 4, 1, 7, 3]
for i in range(len(l)):
    print(l[i])
\end{minted}
그러나 리스트의 원소를 인덱스를 통해 접근하지 않고, 직접 접근할 수도 있습니다.
\begin{minted}{python}
l = [2, 4, 1, 7, 3]
for e in l:
    print(e)
\end{minted}

수학적으로 표현하자면, 전자는 $\sum_{0 \le \texttt{i} < |\texttt{l}|} \texttt{l[i]}$에, 후자는 $\sum_{\texttt{e} \in \texttt{l}} \texttt{e}$에 대응됩니다. 이러한 두 가지 방식은 각각 장단점이 있습니다.
리스트의 원소를 인덱스로 접근할 때에는 해당 원소의 메모리 상 위치를 참조하기 때문에 리스트의 원소를 수정할 수 있습니다.
또한, 리스트의 단조증가 여부 등을 알기 위해서는 \texttt{l[i]}, \texttt{l[i + 1]}와 같이 전후 원소를 비교해야 하며, 이처럼 순서가 중요한 경우 인덱스로 접근하는 것이 바람직합니다.

리스트의 원소를 직접 접근할 때에는 해당 원소의 값을 가지는 변수---위의 예시에서는 \texttt{e}---를 새로 생성하기 때문에, 리스트를 수정하는 것이 아니라 해당 지역 변수의 \texttt{e} 값만 수정이 됩니다.
따라서 다음에 나오는 두 코드는 다른 결과를 내놓게 됩니다.
\begin{minted}{python}
l = [2, 4, 1, 7, 3]
for i in range(len(l)):
    l[i] += 1

print(l)  # [3, 5, 2, 8, 4]
\end{minted}
\begin{minted}{python}
l = [2, 4, 1, 7, 3]
for e in l:
    e += 1

print(l)  # [2, 4, 1, 7, 3]
\end{minted}

파이썬에서는 이 둘을 모두 사용할 수 있게 \pyin{enumerate} 함수를 제공합니다:
\begin{minted}{python}
l = [2, 4, 1, 7, 3]
for i, e in enumerate(l):
    l[i] += e

print(l)  # [4, 8, 2, 14, 6]
\end{minted}

\subsection{\texttt{break}와 \texttt{continue}}
특정 상황에서 반복문을 중단하거나, 바로 다음 반복문 실행으로 넘어가고 싶다면 \texttt{break}와 \texttt{continue}를 활용할 수 있습니다.
아래에 \texttt{break}와 \texttt{continue}을 사용한 예시가 나와 있습니다.
\begin{minted}{python}
for i in range(8):
    if i == 5:
        break
    print(i, end=" ")

# 0 1 2 3 4
\end{minted}
\begin{minted}{python}
for i in range(8):
    if i == [3, 5]:
        continue
    print(i, end=" ")

# 0 1 2 4 6 7
\end{minted}

\texttt{break}와 \texttt{continue}는 \texttt{while}문과 결합하면 효과적으로 사용할 수 있습니다.
일반적으로 \texttt{while}문은 어떤 조건을 만족할 때 종료가 되는데, 반복문 내부에서도 마치고 싶을 때가 있기 때문입니다.

\subsubsection{다중 반복문}
다중 반복문에 \texttt{break}와 \texttt{continue}는 가장 가까이 있는 반복문만을 종료하거나 넘기게 됩니다.
다중 반복문에서의 \texttt{break}와 \texttt{continue}을 나타낸 예시입니다:
\begin{minted}{python}
i = 0
while i < 5:
    j = 0
    while j < 5:
        if i > 0:
            break
        print(j, end=" ")
        j += 1
    print(i, end=" ")
    i += 1

# 0 1 2 3 4 0 1 2 3 4
\end{minted}
\begin{minted}{python}
i = 0
while i < 5:
    j = 0
    while j < 5:
        if i > 0:
            continue
        print(j, end=" ")
        j += 1
    print(i, end=" ")
    i += 1

# infinite loop
\end{minted}
\begin{minted}{python}
for i in range(5):
    if i > 0:
        continue
    j = 0
    while j < 5:
        if i > 0:
            break
        print(j, end=" ")
        j += 1
    print(i, end=" ")

# 0 1 2 3 4 0
\end{minted}

\section{파일 입출력}
많은 양의 자료를 다뤄야 하는 경우, 외부 파일에서 데이터를 읽고 쓰는 것이 필요합니다.

\subsection{파일 입력}
아래와 같이 파일에서 문자열을 읽어올 수 있습니다.
\begin{minted}{python}
fin = open("input.txt", "r")
content = fin.read()
words = content.split()
fin.close()
\end{minted}
첫 번째 줄에서 \pyin{open("input.txt", "r")}을 통하여 \texttt{input.txt} 파일을 읽기 전용(\texttt{"r"})으로 열어 \texttt{fin}에 할당하게 됩니다.
두 번째 줄에서 이 파일을 문자열로 읽어 \texttt{content}에 저장한 후, 세 번째 줄에서는 \texttt{content}의 문자열을 단어별로 끊어 \texttt{words}라는 리스트에 저장을 하게 됩니다.
마지막 줄에서는 이렇게 열린 파일을 닫습니다.
한 번 연 파일을 닫지 않는 것은 리소스 낭비로 이어지기 때문에 반드시 닫아주어야 합니다.\footnote{\texttt{with} 문법을 통해 자동으로 파일을 닫게 할 수도 있습니다.}

이를 활용하여 \texttt{dictionary.txt} 파일의 가장 긴 단어를 최대/최소 패턴을 통해 다음과 같이 찾을 수 있습니다.
\begin{minted}{python}
def longest_word(filename):
    f = open(filename, "r")
    words = f.read().split()
    f.close()

    max_word = None
    max_len = -1

    for word in words:
        if max_len < len(word):
            max_word = word
            max_len = len(word)

    return maxword
\end{minted}
이를 쓰면 \verb/dictionary.txt/에서 pneumonoultramicroscopicsilicovolcanoconiosis라는 단어를 찾아낼 수 있습니다.

이번에는 파일에서 문자열을 읽어들여 처리를 하는 다양한 방법에 대해서 알아보고, 파일에 문자열을 입력하는 법도 알아볼 것입니다.
어떤 문자열에 \pyin{strip()} 메소드를 취하면 문자열 양 끝에 있는 공백 문자들을 없앤 새로운 문자열을 반환합니다.
또한, \pyin{split()} 메소드는 공백으로 구분된 단어들의 리스트를 반환합니다.
\begin{minted}{python}
f = open("./input.txt", "r")
print(type(f))  # <type 'file'>

for i in range(5):
    s = f.readline().strip()
    print(s)

print()

for line in f:
    s = line.strip()
    print(s)

f.close()
\end{minted}
위 코드에서 \texttt{for}문을 통해 바로 \texttt{f}을 줄 단위로 읽어올 수 있다는 것을 확인할 수 있습니다.

\subsubsection{파일 출력}
아래는 구구단을 파일로 작성하는 코드입니다.
\begin{minted}{python}
f = open("./gugu.txt", "w")

for i in range(1, 10):
    for j in range(1, 10):
        f.write(f"{i * j} ")
    f.write("\n")

f.close()
\end{minted}
유의할 점은, \pyin{write}은 \pyin{print}와는 다르게 개행을 자동으로 하지 않는다는 것입니다.
따라서 직접 \pyin{.write("\textbackslash n")}을 통해 개행 문자를 넣어야 합니다.

이외에도 다음과 같은 메소드를 활용할 수 있습니다.
\begin{itemize}
    \item \pyin{.read()}: 전체 내용을 한 문자열로 읽어옵니다.
    \item \pyin{.readlines()}: 각 줄의 내용을 문자열로 가지는 리스트를 반환합니다.
    \item \pyin{.writelines()}: 리스트의 각 원소를 줄로 가지도록 파일에 작성합니다.
\end{itemize}
\end{document}
