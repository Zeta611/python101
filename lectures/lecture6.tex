\documentclass[../main.tex]{subfiles}

\begin{document}
지금까지 반복문의 다양한 패턴을 알아보았으니, 이를 정리하고 \texttt{for} 문을 구성하는 새로운 방법과 \texttt{break}, \texttt{continue} 등의 다양한 기법에 대해 알아볼 것입니다.
나아가 이를 활용하여 다차원 리스트를 다루어 볼 것입니다.
한편으로는 텍스트 파일을 읽고 출력하는 파일 입출력에 대해서 간단히 살펴볼 것이며, 이를 통해 많은 양의 데이터를 처리할 수 있는 길이 열리게 됩니다.

\subsection{Loops반복문 응용}
\subsubsection{range($\cdot$) 함수}
지금까지 세 가지 종류의 반복문 패턴에 대해 알아보았습니다.
이러한 \texttt{for} 문을 구성할 때 지금까지는 \texttt{range($\cdot$)}를 활용하였습니다.
Python 콘솔에 \texttt{range}의 형을 알아보기 위해 \texttt{type(range)}를 쳐보면, 다음과 같은 결과를 얻을 수 있습니다.
\begin{minted}[mathescape,
               linenos,
               breaklines,
               numbersep=5pt,
               frame=lines,
               framesep=2mm]{python}
>>> type(range)
<type 'builtin_function_or_method'>
\end{minted}
또한, 반복문 없이 \texttt{range(n)}의 형을 알아보기 위해 \texttt{type(range(5))}을 콘솔에 입력해봅시다.
\begin{minted}[mathescape,
               linenos,
               breaklines,
               numbersep=5pt,
               frame=lines,
               framesep=2mm]{python}
>>> type(range(5))
<type 'list'>
\end{minted}
이를 통해 \texttt{range($\cdot$)}는 리스트를 반환하는 함수인 것을 알 수 있습니다.

\subsubsection{리스트 원소의 인덱스 접근과 직접 접근}
\texttt{range($\cdot$)}는 리스트를 반환하기 때문에, \texttt{for} 문을 활용할 때 \texttt{range($\cdot$)} 대신에 직접 리스트를 사용하여도 될 것이라고 짐작할 수 있습니다.
지금까지는 리스트 \texttt{l}이 주어졌을 때 \texttt{range(len(l))}을 사용하여 다음과 같이 \texttt{for} 문을 구성하였습니다.
\begin{minted}[mathescape,
               linenos,
               breaklines,
               numbersep=5pt,
               frame=lines,
               framesep=2mm]{python}
l = [2, 4, 1, 7, 3]
for i in range(len(l)):
	print l[i]
\end{minted}
하지만 이제는 \texttt{range($\cdot$)} 대신에 아래와 같이 구성을 할 수 있습니다.
\begin{minted}[mathescape,
               linenos,
               breaklines,
               numbersep=5pt,
               frame=lines,
               framesep=2mm]{python}
l = [2, 4, 1, 7, 3]
for i in [0, 1, 2, 3, 4]:
	print l[i]
\end{minted}
나아가 리스트의 원소를 인덱스를 통해 접근하지 않고, 직접 접근할 수 있습니다.
\begin{minted}[mathescape,
               linenos,
               breaklines,
               numbersep=5pt,
               frame=lines,
               framesep=2mm]{python}
l = [2, 4, 1, 7, 3]
for e in l:
	print e
\end{minted}

수학적으로 표현하자면, 전자는 $\sum_{\texttt{i} \in |\texttt{l}|} \texttt{l[i]}$에, 후자는 $\sum_{\texttt{e} \in \texttt{l}} \texttt{e}$에 대응됩니다. 이러한 두 가지 방식은 각각 장단점이 있습니다.
리스트의 원소를 인덱스로 접근할 때에는 해당 원소의 메모리 상 위치를 참조하기 때문에 리스트의 원소를 수정할 수 있습니다.
또한, 리스트의 단조증가 여부 등을 알기 위해서는 \texttt{l[i]}, \texttt{l[i + 1]}와 같이 전후 원소를 비교해야 하며, 이처럼 순서가 중요한 경우 인덱스로 접근하는 것이 바람직합니다.
리스트의 원소를 직접 접근할 때에는 해당 원소의 값을 가지는 변수--위의 예시에서는 \texttt{e}--를 새로 생성하기 때문에, 리스트를 수정하는 것이 아니라 해당 지역 변수의 \texttt{e} 값만 수정이 됩니다.
또한, \texttt{e}는 반복문을 한 번 수행할 때마다 다음 원소의 값으로 초기화가 됩니다.
따라서, 다음에 나오는 두 코드는 다른 결과를 내놓게 됩니다.
\begin{minted}[mathescape,
               linenos,
               breaklines,
               numbersep=5pt,
               frame=lines,
               framesep=2mm]{python}
l = [2, 4, 1, 7, 3]
for i in range(len(l)):
	l[i] += 1

print l  # [3, 5, 2, 8, 4]
\end{minted}
\begin{minted}[mathescape,
               linenos,
               breaklines,
               numbersep=5pt,
               frame=lines,
               framesep=2mm]{python}
l = [2, 4, 1, 7, 3]
for e in l:
	e += 1

print l  # [2, 4, 1, 7, 3]
\end{minted}
원소를 직접 접근하는 방식은 인덱스를 통한 간접 접근보다는 문법적으로 간결하고 직관적이기 때문에, 순서의 고려가 필요 없을 때는 직접 접근이 바람직합니다.

\subsubsection{\texttt{break}/\texttt{continue}}
특정 상황에서 반복문을 그만 실행하거나, 해당 회만 넘어가고 싶다면 \texttt{break}와 \texttt{continue}를 활용할 수 있습니다.
비록 코드가 복잡한 상황에서 \texttt{break}와 \texttt{continue}를 사용한다면 코드의 구조가 더욱 꼬이고 디버깅이 힘들어지지만, 적절한 상황에서의 활용은 코드 작성의 생산성을 높일 수 있습니다.
이론적으로는 \texttt{break}와 \texttt{continue} 없이 불리언 함수를 활용하여 동일한 기능을 구현할 수 있습니다.
아래에는 \texttt{break}와 \texttt{continue}의 활용 예가 제시되어 있습니다.
\begin{minted}[mathescape,
               linenos,
               breaklines,
               numbersep=5pt,
               frame=lines,
               framesep=2mm]{python}
for i in range(8):
	if i == 5:
		break
	print i,

# 0 1 2 3 4
\end{minted}
\begin{minted}[mathescape,
               linenos,
               breaklines,
               numbersep=5pt,
               frame=lines,
               framesep=2mm]{python}
for i in range(8):
	if i == [3, 5]:
		continue
	print i,

# 0 1 2 4 6 7
\end{minted}

\texttt{break}와 \texttt{continue}는 \texttt{while} 문과 결합하면 효과적으로 사용할 수 있습니다.
일반적으로 \texttt{while} 문은 어떤 조건을 만족할 때 종료가 되는데, 특정 상황에서 반복문을 넘어가거나 중단해야 한다면 사용할 수 있습니다.
또한 다중 반복문의 경우, \texttt{break}와 \texttt{continue}를 감싸고 있는 가장 내부의 반복문만을 종료하거나 넘기게 됩니다.
아래에 다중 반복문에서의 \texttt{break}와 \texttt{continue} 활용과 그에 따른 출력값이 적혀 있습니다.
\begin{minted}[mathescape,
               linenos,
               breaklines,
               numbersep=5pt,
               frame=lines,
               framesep=2mm]{python}
i = 0
while i < 5:
	j = 0
	while j < 5:
		if i > 0:
			break
		print j,
		j += 1
	print i,
	i += 1

# 0 1 2 3 4 0 1 2 3 4
\end{minted}
\begin{minted}[mathescape,
               linenos,
               breaklines,
               numbersep=5pt,
               frame=lines,
               framesep=2mm]{python}
i = 0
while i < 5:
	j = 0
	while j < 5:
		if i > 0:
			continue
		print j,
		j += 1
	print i,
	i += 1

# infinite loop
\end{minted}
\begin{minted}[mathescape,
               linenos,
               breaklines,
               numbersep=5pt,
               frame=lines,
               framesep=2mm]{python}
for i in range(5):
	if i > 0:
		continue
	j = 0
	while j < 5:
		if i > 0:
			break
		print j,
		j += 1
	print i,

# 0 1 2 3 4 0
\end{minted}

\subsection{파일 입출력}
USACO 문제를 해결하거나 많은 양의 자료를 활용해야 하는 경우, 외부 파일에서 데이터를 읽어오는 것이 필요합니다.
아래와 같이 파일에서 문자열을 읽어올 수 있습니다.
\begin{minted}[mathescape,
               linenos,
               breaklines,
               numbersep=5pt,
               frame=lines,
               framesep=2mm]{python}
fin = open("input.txt", "r")
content = fin.read()
words = content.split()
\end{minted}
첫 번째 줄에서 \texttt{open("input.txt", "r")}을 통하여 \texttt{"input.txt"}를 읽기 전용(\texttt{"r"})으로 열어 \texttt{fin}에 배정하게 됩니다.
두 번째 줄에서 이 파일을 문자열로 읽어 \texttt{content}에 저장한 후, 세 번째 줄에서는 \texttt{content}의 문자열을 단어별로 끊어 \texttt{words}라는 리스트에 저장을 하게 됩니다.
혹은, 줄여서 \texttt{words = open("input.txt", "r").read().split()}으로 써도 됩니다.

이를 활용하여 \texttt{dictionary.txt} 파일의 가장 긴 단어를 최대/최소 패턴을 통해 다음과 같이 찾을 수 있습니다.
\begin{minted}[mathescape,
               linenos,
               breaklines,
               numbersep=5pt,
               frame=lines,
               framesep=2mm]{python}
def longest_word(filename):
	words = open(filename, "r").read().split()
	
	maxword = ""
	maxlen = 0
	
	for word in words:
		if maxlen < len(word):
			maxword = word
			maxlen = len(word)

	return maxword
\end{minted}

이번에는 파일에서 문자열을 읽어들여 처리를 하는 다양한 방법에 대해서 알아보고, 파일에 문자열을 입력하는 법도 알아볼 것입니다.
뒤에서 자세히 배우겠지만, 어떤 문자열에 \texttt{.strip()} 메소드를 취하면 문자열 양 끝에 있는 공백 문자들을 없앤 새로운 문자열을 반환합니다.
또한, \texttt{.split()} 메소드는 공백으로 구분된 단어들의 리스트를 반환합니다.
\begin{minted}[mathescape,
               linenos,
               breaklines,
               numbersep=5pt,
               frame=lines,
               framesep=2mm]{python}
f = open("./input.txt", "r")
print type(f)  # <type 'file'>

for i in range(5):
	s = f.readline().strip()
	print s

print

for line in f:
	s = line.strip()
	print s
\end{minted}
위 코드에서 \texttt{for} 문에 바로 파일 \texttt{f}를 사용할 수 있다는 것을 보았습니다.
이는 \texttt{for} 문에서 파일을 사용하면 자동으로 \texttt{readline()}을 불러오기 때문입니다.

파일에 내용을 작성하는 예시는 다음에 주어져 있습니다.
아래는 구구단을 입력하는 코드입니다.
\begin{minted}[mathescape,
               linenos,
               breaklines,
               numbersep=5pt,
               frame=lines,
               framesep=2mm]{python}
f = open("./multiply.txt", "w")

for i in range(1, 10):
	for j in range(1, 10):
		s = str(i * j) + " "
		f.write(s)
	f.write("\n")

f.close()
\end{minted}
유의할 점은, \texttt{.write($\cdot$)}은 \texttt{print}와는 다르게 개행을 자동으로 하지 않는다는 것입니다.
따라서 직접 \texttt{.write("\textbackslash n")}을 통해 개행 문자를 넣어야 합니다.
(USACO에서는 파일 출력을 할 때 맨 마지막에 개행 문자를 넣어줘야 합니다.)
또한, 항상 파일을 열면 \texttt{.close()}를 통해 닫아줘야 합니다.

이외에도 다음과 같은 메소드를 활용할 수 있습니다.
\begin{itemize}
	\item \texttt{.read()}: 전체 내용을 하나의 문자열로 가져옵니다.
	\item \texttt{.readlines()}: 각 줄의 내용을 문자열로 가지는 리스트를 반환합니다.
	\item \texttt{.writelines()}: 리스트의 각 원소를 줄로 가지도록 파일에 작성합니다.
\end{itemize}
따라서, 위의 \texttt{.read()}와 문자열의 \texttt{.split()} 메소드를 통해 다음과 같이 파일을 읽을 수 있습니다.
\begin{minted}[mathescape,
               linenos,
               breaklines,
               numbersep=5pt,
               frame=lines,
               framesep=2mm]{python}
f = open("./input.txt", "w")
s = f.read().split()

for word in s:
	print word
\end{minted}


\subsection{예제}
\begin{enumerate}
\item Write a function \texttt{occurrences\_integers(filename)} that writes a file \texttt{occurences.out} with increasing order of numbers in \texttt{filename} along with their occurrences.
For instance, a file with the following content as an input:
\begin{verbatim}
5 4 10 10 4
\end{verbatim}
should write a file with:
\begin{verbatim}
4 2
5 1
10 2
\end{verbatim}

\item Write a function \texttt{sieve\_eratosthenes(n)} that writes a file \texttt{primes.out} with all primes numbers less than or equal to \texttt{n} using seive of Eratosthenes.
Create a list using \texttt{range(2, n + 1)} and use \texttt{.pop($\cdot$)} to remove composite numbers.
An input $13$ should write a file:
\begin{verbatim}
2 3 5 7 11 13
\end{verbatim}
\end{enumerate}
\end{document}