\documentclass[../main.tex]{subfiles}

\begin{document}
지난 시간에는 미리 정해진 동선을 따라 휴보가 이동하도록 조건문과 반복문 등을 사용하여 코드를 작성하는 법을 알아보았습니다.
주어진 동선을 군수열 분석하듯이 잘라 루프를 구성하고, beeper를 놓거나 줍는 방식이었습니다.
Beeper를 줍는 것은 beeper가 해당 격자점에 있을 경우(\texttt{on\_beeper()}) 판단하여 줍고, 놓는 것도 현재 휴보가 beeper를 가지고 있다면(\texttt{carries\_beeper()}) 자동으로 놓게 하였습니다.
그렇다면 특정 조건을 만족시킬 때 휴보가 자동으로 이동하는 방법이 있을지 궁금할 것입니다.
예컨대 휴보가 알아서 미로를 탐색하는 작업을 위해서는 벽의 위치와 같은 주변 환경에 따라 자동으로 휴보가 이동할 수 있는 코드가 필요합니다.
이를 위해 이번 시간에는 \texttt{front\_is\_clear()}, \texttt{left\_is\_clear()}, 그리고 \texttt{right\_is\_clear()}와 같이 휴보의 앞과 옆에 장애물이 있는지 판단하는 메소드를 배울 것입니다.
\begin{figure}[H]
\centering
\includegraphics[width=0.5\linewidth]{"./lectures/lecture7_maze"}
\label{fig:lecture7maze}
\end{figure}

조금 원론적으로는, 현재 휴보가 주변 환경을 감지하여(sense) 이를 바탕으로 동선을 결정한 후(control) 실행에 옮기는 과정(actuate)은 일반적인 자율 시스템 전반에 적용되는 과정입니다.
집에서 사용하는 로봇 청소기나 무인 드론, 자율 주행 자동차 등 모두 sensor--controller--actuator의 과정을 따름을 알 수 있습니다.
토이 로봇을 다룰 때에는 beeper의 유무, 벽의 유무를 파악하는 메소드가 sensor에 해당하고, 우리가 작성하는 코드가 controller, 그리고 토이 로봇을 움직이게 하는 메소드가 actuator에 해당합니다.

\section{자율 이동}
지난 번에 아래와 같이 beeper를 줍는 코드를 보았습니다.
\begin{minted}[mathescape,
               linenos,
               breaklines,
               numbersep=5pt,
               frame=lines,
               framesep=2mm]{python}
for i in range(9):
    hubo.move()
    if hubo.on_beeper():
        hubo.pick_beeper()
\end{minted}
이를 beeper가 놓여 있는 만큼 주을 수 있도록 할 수 있게 \texttt{while} 반복문을 통해서 아래와 같이 바꾸었습니다.
\begin{minted}[mathescape,
               linenos,
               breaklines,
               numbersep=5pt,
               frame=lines,
               framesep=2mm]{python}
for i in range(9):
    hubo.move()
    while hubo.on_beeper():
        hubo.pick_beeper()
\end{minted}
나아가 휴보가 앞으로 나아갈 수 있는 만큼 이동하면서 휴보가 이동하도록 다음과 같이 코드를 수정할 수 있습니다.
\begin{minted}[mathescape,
               linenos,
               breaklines,
               numbersep=5pt,
               frame=lines,
               framesep=2mm]{python}
while hubo.front_is_clear():
    hubo.move()
    while hubo.on_beeper():
        hubo.pick_beeper()
\end{minted}
이렇게 코드를 수정하면 너비가 10인 공간 뿐만이 아니라 임의의 너비를 가진 공간에서 beeper를 줍도록 할 수 있습니다.
이제 \texttt{front\_is\_clear()} 메소드가 어떤 역할을 가지는 대략적인 감이 올 것입니다.
\texttt{front\_is\_clear()} 메소드는 \texttt{on\_beeper()} 등과 같은 불리언 함수로, 휴보 앞에 벽이 있는지를 파악합니다.
만약 휴보 앞에 벽이 있어서 나아갈 수 없다면 \texttt{False}를, 있다면 \texttt{True}를 반환합니다.
\texttt{front\_is\_clear()}만으로도 휴보 양 옆에 장애물이 있는지 판단할 수는 있습니다.
\texttt{turn\_right()}과 같이 방향 전환을 하는 메소드와 결합하여 사용하는 방법입니다.
하지만 \texttt{cs1\_robots} 패키지에는 이런 번거로움을 피하기 위해 \texttt{front\_is\_clear()} 메소드와 더불어 \texttt{right\_is\_clear()}과 \texttt{left\_is\_clear()} 메소드도 함께 준비되어 있습니다.
각각 우측과 좌측에 벽이 있는지 판단하는 메소드입니다.

이제 휴보가 자동으로 움직일 모든 준비가 끝났습니다.
지금까지 배운 sensor와 관련된 메소드와 actuator와 관련된 메소드를 통해서 휴보가 자동으로 움직이도록 코드를 작성--즉 control--할 수 있습니다.
여러 문제를 풀어보기에 앞서, 휴보가 주어진 공간의 외곽 모서리를 짚으며 이동하게 하는 코드를 작성하는 법을 차근차근 알아봅시다.
이 때 휴보가 $(1, 1)$에서 반시계 방향으로 돌아간다고 합시다.

\textbf{전체적인 구조 설계하기}---일단 코드를 작성하기에 앞서, 휴보가 어떤 방식으로 주어진 환경을 돌 수 있을지 대략적으로 생각해보아야 합니다.
현재 다루는 환경은 직각의 벽으로만 공간으로만 이루어져 있으므로, 일단 앞으로 쭉 이동하다가 벽이 나오면 회전을 하면 될 것입니다.
지형이 구체적으로 어떻게 생겼는지 모르기 때문에 \texttt{while} 반복문을 구성하여 조건에 따라 특정 명령을 반복 수행하도록 코드를 작성해야 합니다.
지금까지는 다음과 같이 코드를 작성할 수 있습니다.
\begin{minted}[mathescape,
               linenos,
               breaklines,
               numbersep=5pt,
               frame=lines,
               framesep=2mm]{python}
while #조건:
    # 이동하다가 벽이 나오면 회전
\end{minted}

\textbf{종료 조건 생각하기}---코드가 어떤 조건에서 terminate종료될 것인지 생각하는 것이 매우 중요합니다.
자동으로 휴보가 움직이므로 \texttt{while} 문 등으로 코드가 반복 실행될 것이기 때문입니다.
지난 시간에 이러한 문제를 다룰 때는 이동한 거리를 카운터 패턴으로 세어 해결했습니다.
이 아이디어를 확장한다면 $x$ 방향 움직임과 $y$ 방향 움직임의 합이 0이 되면 닫힌 루프를 구성하여 시점에 도달했음을 판단할 수 있습니다.
Beeper를 활용한 좀 더 단순한 접근법도 생각할 수 있습니다.
처음 시작할 때 beeper를 내려놓고 해당 beeper 위에 돌아왔을 때 놓은 beeper를 다시 주은 후 \texttt{while} 문을 종료하면 됩니다.
\begin{minted}[mathescape,
               linenos,
               breaklines,
               numbersep=5pt,
               frame=lines,
               framesep=2mm]{python}
hubo.drop_beeper()
hubo.move()
while not hubo.on_beeper():
    # 이동하다가 벽이 나오면 회전
hubo.pick_beeper()
\end{minted}

\textbf{간단한 경우 생각하기}---처음부터 완벽한 코드를 작성할 필요는 없습니다.
일단 직사각형 형태의 단순한 지형을 다루어 봅시다.
$(1, 1)$에서 반시계로 이동하므로 벽이 나왔을 경우 좌회전만 하면 됩니다.
따라서 다음과 같이 어렵지 않게 코드를 작성할 수 있습니다.
\begin{minted}[mathescape,
               linenos,
               breaklines,
               numbersep=5pt,
               frame=lines,
               framesep=2mm]{python}
hubo.drop_beeper()
hubo.move()
while not hubo.on_beeper():
    if hubo.front_is_clear():
        hubo.move()
    else:  # 좌회전을 해야 하는 경우
        hubo.turn_right()
hubo.pick_beeper()
\end{minted}

\textbf{예외의 경우 생각하여 일반화하기}---이제 직사각형 형태의 지형에서는 잘 작동하는 코드를 완성하였습니다.
하지만 지형이 직사각형이라는 보장이 없습니다.
다음과 같은 지형을 생각할 수 있기 때문입니다.

\begin{figure}[H]
\centering
\includegraphics[width=0.5\linewidth]{"./lectures/lecture7_exception"}
\label{fig:lecture7exception}
\end{figure}

이러한 경우, 기존 코드에서 어떤 부분을 수정해야 하는지 파악해야 합니다.
일단 문제가 되는 빨간 동그라미 부분을 처리해야 합니다.
이는 앞에는 장애물이 없음에도 불구하고 직진이 아니라 우회전을 해야 하는 부분입니다.
하지만 현재 코드에서는 줄 4에서 이 경우를 걸러내지 못하고 무조건 직진을 하게 하는 것이 문제가 됩니다.
코드를 어떻게 바꿔야 할지 생각을 하기 위해서는, 우회전을 해야 하는 경우와 기존처럼 직진을 하는 경우의 차이만을 생각하면 됩니다.
구조적으로 생각하자면 휴보가 앞으로 직진을 하는 행동이 줄 5에 나와 있으므로, 이에 대한 조건식인 줄 4 이전에 우회전을 하는 경우를 고려하는 조건식을 넣으면 됩니다.
이러한 경우, 가능의 여부를 생각하는 것이 도움이 될 수 있습니다.
종종 가능의 여부와 필요의 여부가 일치하기 때문입니다.
완전히 일치하지 않더라도 조건식을 생각하는데 도움이 될 수 있습니다.
이 문제는 우회전을 할 수 있는지의 여부와 우회전을 해야 하는지의 여부가 일치합니다.
오른쪽이 비어 있어야 우회전을 할 수 있는데, 이 경우가 바로 우회전을 해야 하는 상황입니다.
따라서 아래와 같이 코드를 수정할 수 있습니다.

\begin{minted}[mathescape,
               linenos,
               breaklines,
               numbersep=5pt,
               frame=lines,
               framesep=2mm]{python}
hubo.drop_beeper()
hubo.move()
while not hubo.on_beeper():
    if hubo.right_is_clear():
        hubo.turn_right()
    elif hubo.front_is_clear():
        hubo.move()
    else:
        hubo.turn_right()
hubo.pick_beeper()
\end{minted}

결국 위 코드를 잘 보면 미로의 전형적인 해법인 ``벽 짚고 따라가기''인 것을 알 수 있습니다.
하지만 처음부터 벽을 짚고 따라가는 아이디어를 코드로 구현하려고 하는 것보다는, 지금까지 한 단계씩 밟아온 것처럼 코드를 점진적으로 수정 및 보완하는 것이 더 쉽고 실수를 줄이는 방법입니다.

\end{document}
